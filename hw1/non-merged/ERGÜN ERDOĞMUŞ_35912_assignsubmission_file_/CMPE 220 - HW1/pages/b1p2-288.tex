
\documentclass[11pt]{amsbook}

\usepackage{../HBSuerDemir}

\begin{document}
    % ++++++++++++++++++++++++++++++++++++++
    \hPage{b1p2/288}
    % ++++++++++++++++++++++++++++++++++++++
    
    After having one of the values of $t = \tan \theta$ (preferably the positive one) the values of $\cos \theta$ and $\sin \theta$ can be obtained by the use of a right triangle with legs $t$ and $1$ \par
    
    To obtain $\cos\theta$ and $\sin\theta$ one may proceed in a different way as follows:
    
    
    \[\cos(2\theta) = \frac{1}{\sqrt{1 + \tan^2(2\theta)}} \implies \begin{cases} 
    
        \cos\theta = \sqrt{\frac{1 + \cos2}{2}} \\
        \sin\theta = \sqrt{\frac{1 - \cos2}{2}}
    
    \end{cases} \]
    
    
    \noindent where positive signs are taken for convenience. \par
    \underline{Examples.} \par
        
    \begin{hEnumerateArabic}
        \item Given the line $y = \sqrt{3}x + 2$, rotate the coordinate axes by the angle $\pi/3$ and obtain the new equation of the line.
        \item Given the parabola $y = x^2$ and the line $y = x - 3$, find the new equation of the parabola after the rotation of the coordinate axes such that x'-axis be parallel to the given line.
        \item Given $x^2 - 3xy + y^2 - y = 0$, apply rotation to eliminate the cross term and obtain the new equation of the conic.\\
    \end{hEnumerateArabic}
    
    \underline{Solution.} \par

    \begin{hEnumerateArabic}
        \item The transforming formulas being 
        
        \[x = x'\cos(\frac{\pi}{3}) - y'\sin(\frac{\pi}{3}) = \frac{1}{2}x' - \frac{\sqrt{3}}{2}y' \]
        
        \[y = x'\sin(\frac{\pi}{3}) + y'\cos(\frac{\pi}{3}) = \frac{\sqrt{3}}{2}x' +  \frac{1}{2}y' \]
        
        \noindent we have
        
        \[y  = \sqrt{3}x + 2 \implies \frac{\sqrt{3}}{2}x' + \frac{1}{2}y' = \sqrt{3}(\frac{1}{2}x' - \frac{\sqrt{3}}{2}y') + 2 \implies y' = 1 \]
        
        \item The angle of rotation is $\pi/4$. Then the transforming formulas are 
        \[x = \frac{1}{\sqrt{2}} (x' - y') \]
        \[y = \frac{1}{\sqrt{2}} (x' + y') \]
        
    \end{hEnumerateArabic}
\end{document}  

