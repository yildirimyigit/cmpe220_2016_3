\documentclass[11pt]{amsbook}

\usepackage{../HBSuerDemir}


\begin{document}    	

\chapter{INTEGRATION}
\section{INDEFINITE INTEGRAL (PRIMITIVE FUNCTION)}

\hPage{b1p2/341}

\subsection{DEFINITIONS}
$F(X)$ is called a \hDefined{primitive function} or simply a \hDefined{primitive} of the function $f(x)$ is the derivative of $F(x)$
\par According to the definiton, one can write

\begin{align*}
	D F(x) = f(x) \Longrightarrow F(x) = \frac{1}{D}f(x)
\end{align*}

\noindent where $\frac{1}{D} (= D^{-1})$ is the inverse of the derivative otorator D and accordingly $F(x)$ is said to be an \hDefined{antiderivative} of $f(x)$.
\par In terms of differentials we have

\begin{align*}
	D F(x) = f(x) \Longrightarrow dF(x) = f(x)dx \Longrightarrow F(x) = \frac{1}{d} (f(x)dx)
\end{align*}

where $\frac{1}{d}(=d^{-1})$ is the inverse of the differential operator d, denoted byte symbol $\int$ which is called \hDefined{integral sign}. Then

\begin{align*}
	F(x) = \int f(x)dx \quad \text{(Read: \hDefined{indefinite integral} of $f(x)dx)$}
\end{align*}

\par The following theorem justifies the term "indefinite":

\begin{thm}
	Any two primitives of a given function differ by a constant.
	\begin{proof}
		Let $F(x)$,  $G(x)$ be any two primitives of the given function f(x).
		\begin{align*}
			D &F(x) = D G(x) = f(x) \Longrightarrow D(F(x) - G(x)) = 0 \\
			\Longrightarrow \quad &F(x) - G(x) = c\quad \text{(c is an arbitrary constant).}
		\end{align*}
	\end{proof}
\end{thm}

\begin{cor}
	If $F(x)$ is a primitive of $f(x)$, then any primitive is in the form $F(x) + c$ where $c \in R$, namely

	\begin{align*}
		\int f(x)dx + c
	\end{align*}

	where c is called the \hDefined{constant of integration}, and f(x) the \footnote{This corollary ends in the next page.}
\end{cor}

\end{document}