\documentclass{amsbook}

\usepackage[pdftex]{graphicx}
\usepackage{amsmath}
\usepackage{../HBSuerDemir}

\begin{document}
\underline{Solution}.\\
\begin{align*}
D&=6\cdot(-1)^{1+3}
\begin{vmatrix}
  3 & 5 \\
  4 & 9
\end{vmatrix}
+7\cdot(-1)^{2+3}
\begin{vmatrix}
  8 & 1 \\
  4 & 9
\end{vmatrix}
+2\cdot(-1)^{3+3}
\begin{vmatrix}
  8 & 1 \\
  3 & 5
\end{vmatrix}\\
&=6(27-20)-7(72-4)+2(40-3)\\
&=42-476+74=116-476=-360.
\end{align*} \footnote{There was a redundant and wrong operation.}
\begin{align*}
D&=3\cdot(-1)^{2+1}
\begin{vmatrix}
  1 & 6 \\
  9 & 2
\end{vmatrix}
+5\cdot(-1)^{2+2}
\begin{vmatrix}
  8 & 6 \\
  4 & 2
\end{vmatrix}
+7\cdot(-1)^{2+3}
\begin{vmatrix}
  8 & 1 \\
  4 & 9
\end{vmatrix}\\
&=-3(2-54)+5(16-24)-7(72-4)\\
&=156-40-476=-360.
\end{align*}
\par Any determinant can be evaluated this way vy expanding it with respect to any row (column), but as the order gets higher, calculations become laborious. The following theorems on determinants are helpful in simplifying the computations.\\
\begin{enumerate}[label=(\Alph*)]
\setcounter{enumi}{2}
\item  THEOREMS ON DETERMINANTS
\end{enumerate}
\par \underline{Theorem 1}. If D is a determinant and $D^{\top}$ is its transpose, then $D^{\top} = D$.\\
\par This is a consequence of evaluation of determinant by (2) and (3).\\
\par \underline{Theorem 2}. If two rows (columns) of a determinant are interchanged, the determinant is changed in sign only.\\
\par When the given determinant is\\
\begin{align*}
D=
\begin{vmatrix}
 \cdots & a_{1k} & \cdots & a_{1r} & \cdots \\\\\\
 \cdots & a_{nk} & \cdots & a_{nr} & \cdots 
\end{vmatrix}
,then\quad D'=
\begin{vmatrix}
 \cdots & a_{1r} & \cdots & a_{1k} & \cdots \\\\\\
 \cdots & a_{nr} & \cdots & a_{nk} & \cdots
\end{vmatrix}
 = -D.
\end{align*}\\
\par This can be proved by induction.
\end{document}  