\documentclass{book}
\usepackage{../HBSuerDemir}

\begin{document}
% ++++++++++++++++++++++++++++++++++++++
\hPage{b2p1/14}
% ++++++++++++++++++++++++++++++++++++++
\begin{proof}
\item
\begin{hEnumerateAlpha}
\item
\begin{align*}
G_{n} & = \sqrt[n]{\left(\frac{2}{1}\right)^{1} \left(\frac{3}{2}\right)^{2} \cdots \left(\frac{n}{n-1}\right)^{n-1} \left(\frac{n+1}{n}\right)^{n}}\\ & =  \sqrt[n]{\frac{1}{1} \cdot \frac{1}{2} \cdots \frac{1}{n-1} \frac{(n+1)^{n}}{n}} = \frac{n+1}{\sqrt[n]{n!}} \rightarrow e.
\end{align*}
$$
\frac{n}{\sqrt[n]{n!} }= \frac{n+1}{\sqrt[n]{n!}} \frac{n}{n+1} \rightarrow e\cdot1 = e
$$
\item
We use the lemma:
\begin{lem}
${{a_{n}} \to 0} \implies {{a_{n}^{n}} \to 0}$.
\end{lem}
Indeed, since ${a_{n}} \to 0$ given $\varepsilon>0$ there is $N>0$ such that$\hAbs{a_n} = \hAbs{a_n - 0}<\varepsilon$ for all $n>N$. Taking $\varepsilon<1$,
$$
\hAbs{a_{n}^{n}}<\varepsilon^n<\varepsilon \implies a_{n}^{n}  \to 0.
$$
Now,
$$
\frac{e}{\sqrt[n]{n!}}=\frac{e/n}{\sqrt[n]{n!}/n} \rightarrow \frac{0}{1/e} - 0 \implies \frac{e^n}{n!}\rightarrow0.
$$


\end{hEnumerateAlpha}
\end{proof}

\subsection{Exercises}
\begin{exercise}
Which ones of the following are subsequences of $(1-(-1)^n)_1$?
\begin{center}
	\begin{tabular}{llll}
		a) $(1)$ \quad  \quad
		&b) $(-1)$ \quad  \quad 
		&c) $(0)$ \quad  \quad
		&d) $(2)$ \quad  \quad
	\end{tabular}\\
	\end{center}
\end{exercise}

\begin{exercise}
\label{ex:01:01:002}
Write two subsequences of
\end{exercise}


\end{document}