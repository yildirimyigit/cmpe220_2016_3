\documentclass{book}
\usepackage{../HBSuerDemir}
\begin{document}

\[
    k(a+ib) = ka+ikb
\]
\underline{Example. Simplify}

\begin{enumerate}
    \item 
    
    $    u = (2 - 3i) - 2(4 + 2i)    $
    
    \item
     
    $    v = \overline{2(3 - 2i) + 3i} $
        
\end{enumerate}

\underline{Solution.}

\begin{enumerate}
    \item
    
    $ u = 2 - 3 i - 8 - 4 i = 2 - 8 - ( 3 i + 4 i ) = - 6 - 7 i $
    
    \item
    
    $ v = \overline{6 - 4i + 3i} = \overline{6 - i} = 6 + i $
\end{enumerate}

    \underline{Multiplication and division}: The product of two complex numbers is obtained as follows$\colon $
    
    \begin{align*} 
      3 . ( a + i b ) ( c - 1 id ) = ac + iad + ibc - i^{2}bd 
      \\= ac + i(ad + be) - bd(Note that
      \\i^{2} = -1)
      \\=(ac - bd) + i(ad + bc) 
    \end{align*}
    \underline{Corollary}. $z = a + ib \longrightarrow z\overline{z} = a^{2} + b^{2} $ 
\\    \underline{Example}. Perform multiplications:
\begin{enumerate}
    \item
    
    $u = (2 - 3i)(5 + i)$
    
    \item
    
    $v =(2 - 3i)(2 + 3i)$

\end{enumerate}
    \underline{Solution}.
\begin{enumerate}
    \item
    $ 10 + 2i- 15i- 3i^{2} = 10 - 13i + 3 = 13 - 13i $
    \item   
    $ v = ( 2 - 3 i ) ( 2 + 3 i ) = 2^{2} + 3^{2} = 4 + 9 = 13 $
\end{enumerate}
    In view of above Corollary, division $u\div v $ is carried out by multiplying the numerator and denominator by the conjugate $\overline{v}$ of the denominator $\colon$ 
        \[
        u \div v =( u \div v )(\overline{v} \div \overline{v}) = (1 \div (v \overline{v})) u\overline{v}
        \]
        
\end{document}
