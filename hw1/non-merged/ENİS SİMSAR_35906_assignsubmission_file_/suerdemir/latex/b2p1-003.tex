\documentclass[11pt]{amsbook}
\usepackage{../HBSuerDemir}

\begin{document}
\hPage{b2p1/003}

\par
\underline{\textbf{Determination of Sequences By Recurrence Relations:}}\\

\par
A sequence $(a_{n})_{p}$ can be defined more generally by a recurrence relation\\

\begin{align*}
	f(a_{n},\, \dotso , a_{n+k}) = 0\\
\end{align*}

\noindent and k consecutive terms $a_{p} , \dotso  , a_{p+k-1}$.\\

\par
The following are two examples for $k = 1$ and $k = 2$\\


\begin{exmp} 
	Given the sequence defined by\\
	\begin{align*}
		a_{1} = 3, \quad and\quad a_{n} = a_{n-1} + 2\\
	\end{align*}
	\begin{hEnumerateAlpha}
		\item obtain the first four terms,\\
		\item find the general term.\\
	\end{hEnumerateAlpha}
	\begin{hSolution}
		\begin{hEnumerateAlpha}
			\item $a_1 = 3$, $a_2 = a_1+2 = 5$, $a_3 = a_2 + 2 =7$, $a_4 = 7 + 2 = 9$ \\
			\item Writing the relation for n = 2, 3, \dotso up to n; and adding these member to member, the intermediate terms $a_2$, \dotso , $a_{n-1}$ are canceled, and $a_n$  is obtained:\\
		\end{hEnumerateAlpha}
		\begin{align*}
			\bcancel{ a_{2}} &= a_1 + 2\\
			\bcancel{ a_{3}} &= \bcancel{ a_{2}} + 2\\
			\vdots &\qquad \quad \vdots \\
			a_{n} &= \bcancel{ a_{n-1}} + 2\\
			\cline{1-2}
			a_n &= a_1 + (n-1)\cdot 2\\
			&= 3 + 2n - 2 = 2n+1
		\end{align*}
	\end{hSolution}
\end{exmp}

\end{document}