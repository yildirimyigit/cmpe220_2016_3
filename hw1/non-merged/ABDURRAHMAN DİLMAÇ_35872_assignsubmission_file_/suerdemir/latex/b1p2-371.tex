\documentclass[11pt]{amsbook}

\usepackage{../HBSuerDemir}

\begin{document}
% ++++++++++++++++++++++++++++++++++++++
\hPage{b1p2/371}
% ++++++++++++++++++++++++++++++++++++++
\begin{thm}
	\footnote{proof of theorem continues from the previous page, 'THEOREM' and 'PROOF' words are not undesirable in my page.}
	\begin{proof}
		will be $-h,\:h$ so that
		\begin{align*}
			A &= \int_{-h}^{h} \! (\alpha x^{2} + \beta x + \gamma)\, \hDif x = (\frac{\alpha}{3}x^{3} + \frac{\beta}{2}x^{2} + \gamma x)_{-h}^{h} \\
			&= \frac{2}{3} \alpha h^{3} + 2 \gamma h = \frac{h}{3} (2 \alpha h^{2} + 6\gamma)
		\end{align*}
		Since,
		\begin{align*}
			y_{0} &= \alpha h^{2} - \beta h + \gamma \\
			4y_{1} &= \qquad\qquad\quad 4\gamma \\
			y_{2} &= \alpha h^{2} + \beta h + \gamma \\
			\cline{1-2}
			y_{0}+4y_{1}+y_{2} &= 2\alpha h^{2} \qquad +6\gamma
		\end{align*}
		we have our result.
		\par Now partitioning $\hPairingParan{a,b}$ regularly for an even number $n$ and applying the above lemma for consecutive pairs of strips and adding the results of each pair, we have
		\begin{align*}
			\frac{h}{3} ((y_{0}+4y_{1}+y_{2})+(y_{2}+4y_{3}+y_{4})+\cdots +(y_{n-2}+4y_{n-1}+y_{n}))
		\end{align*}
		and
		\begin{align*}
			\int_{a}^{b} \! f(x)\, \hDif x = \frac{h}{3}(y_{0}+4y_{1}+2y_{2}+4y_{3}+\cdots +2y_{n-2}+4y_{n-1}+y_{n})
		\end{align*}
		where $h=(b-a)/n$ and $n$ is an even number.
		\par Observe that coefficients of $y_{1}$ are $1$ for $i=0$ and $i=n$; for others, $4$ for odd $i$ and $2$ for even $i$.
	\end{proof}
\end{thm}

\begin{exmp}
	Evaluate the definite integral
	\begin{align*}
		A = \int_{1}^{3}\frac{\hDif x}{x}
	\end{align*}
	approximately (numerically) using the three rules, taking $n=6$.
	\begin{hSolution}
		\footnote{solution of the example continues to the next page. In order not to get errors while compiling, I closed my tags here.}
		We have $h=\frac{3-1}{6}=\frac{1}{3}$ and
		\begin{align*}
			\begin{tabular}{c|cccccccc|}
				$x_{i}$ & $1$ & $\frac{4}{3}$ & $\frac{5}{3}$ & $2$ & $\frac{7}{3}$ & $\frac{8}{3}$ & $3$ \\
				\hline
				$y_{i}$ & $1$ & $\frac{3}{4}$ & $\frac{3}{5}$ & $\frac{1}{2}$ & $\frac{3}{7}$ & $\frac{3}{8}$ & $\frac{1}{3}$
			\end{tabular}
		\end{align*}
	\end{hSolution}
\end{exmp}
\end{document}