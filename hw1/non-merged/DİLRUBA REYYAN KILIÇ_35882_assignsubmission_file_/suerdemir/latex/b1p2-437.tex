\documentclass{article}
\usepackage[utf8]{inputenc}
\pagestyle{empty}

\title{Cmpe220}
\author{dilruba.kilic }
\date{October 2016}

\usepackage{mathtools}
\usepackage{amsmath}
\usepackage{../HBSuerDemir}	% ------------------------
\footnote{I used package mathtools}


\begin{document}
% ++++++++++++++++++++++++++++++++++++++
\hPage{b1p2/437}
% ++++++++++++++++++++++++++++++++++++++

\noindent quickly.
\par \underline{Example 2}. Decompose the following into partial fractions.

$$
\frac{-4x^2 +x -1}{(x^2 -x)(x^2 + 1)^2}
$$

\par \underline{Solution}.It is a proper fraction with denominator
$$
x(x-1)(x^2 +1)^2
$$
Hence by the theorem it has the decomposition
$$
\frac{-4x^2 +x -1}{x(x-1)(x^2 +1)} =
\frac{A}{x} \ \frac{B}{x-1} \ \frac{Cx+D}{x^2 + 1} + \frac{Ex+F}{(x^2 +1)^2}
$$
which when cleared of denominators gives the identity
$$      
-4x^2 +x -1 = ((A+B)x -A)(x^2 +1)^2 +x(x-1)\Big((Cx+D)(x^2+1)+Fx +F\Big)
$$
Now,
\begin{equation*}
  \begin{cases}
    $x = 0:$ \hspace{4mm} $-1=-A$\\
    $x = 1:$ \hspace{4mm} $-4=B$
  \end{cases}
  \qquad\implies\qquad
    A =1,\hspace{2mm}B=-1
\end{equation*}



\par Since\hspace{4mm}$i= \sqrt{-1}$\hspace{2mm}$(i^2 = -1)$\hspace{4mm}is a root if\hspace{4mm}$x^2 +1 = 0$.

\begin{equation*}
    \begin{split}
         x = i: -4i^2 +i -1 &= 0 +i(i-1)(0+ Ei+ F)\\
        \Rightarrow \hspace{5mm} 4 + i - 1 &= (-1 - i)(F + iE)\\
        3 + i &= (E-F) - (E + F)i
    \end{split}
\end{equation*}
implying $E - F = 3$,\hspace{3mm}$E+F =-1 \hspace{4mm}\Rightarrow \hspace{4mm} E=1,  F=-2$. 
Then the identity \newline
identitity becomes
$$
-4x^2 +x -1 =  -(x^2 +1)^2 + x(x-1)\Big((Cx + D)(x^2 +1)+(x-2)\Big)
$$
$x = -1: \hspace{3mm} -6 = -4 +2\Big((-C +D)2 -3\Big) \hspace{3mm} \Rightarrow \hspace{3mm} C - D = -1$
\newline
$x = 2: \hspace{3mm} -15 = -25 + 2(2C + D)5 \hspace{14mm} \Rightarrow \hspace{3mm} 2C + D = 1$
\medskip
\newline
Hence, $C=0$, $D=1$, and
$$
\frac{-4x^2 +x -1}{x(x-1)(x^2 +1)} = \frac{1}{x} - \frac{1}{x-1} + \frac{1}{x^2 + 1} + \frac{x-2}{(x^2 +1)^2}
$$

Therefore the integrals of proper rational functions are
\end{document}
