\documentclass{article}
\usepackage{../HBSuerDemir}
\usepackage{gauss}			%for the matrix row operations.

%This document covers the page 89 in the first part of the second calculus book of B. Süer & H.Demir.


\begin{document}
\hPage{b2p1/89}

An elementary operation followed by another is an operation called a \textcolor{red}{\textit{row operation}}. The row operation
\[
	R_{i} + \lambda{R_{j}} \quad (\lambda \in \hSoR)
\]
has an obvious meaning.

A finite number of elementary operations applied to a matrix
$A$ produce a matrix $A'$ which is said to be a \textcolor{red}{\textit{row equivalent}} to $A$,
written
\[
	A \sim A'
\]

In transforming a matrix $A$ to a row equivalent matrix $A'$ useful notations for $R_{i} \leftrightarrow R_{j}$ and $\lambda R_{k} + R_{j}$ are

%================
%matrix row(and/or column) operation using gauss package.
\begin{center}
	%[b] stands bmatrix. similarly one can use [v],[p] and so on.
	\[\begin{gmatrix}[b]		
		\cdots & \cdots  & \cdots \\
		a_{i1} & \cdots  & a_{in} \\
		\cdots & \cdots  & \cdots \\
		a_{j1} & \cdots  & a_{jn} \\
		\cdots & \cdots  & \cdots \\
		a_{k1} & \cdots  & a_{kn}
	%takes the row operations.
	\rowops	
		%swap second row and fourth row.
		\swap{1}{3}
		%multiply sixth row with lambda and add to the fourth row.		
		\add[\lambda]{5}{3}		
	\end{gmatrix}\]
\end{center}
\footnote{Gauss package is used to demonstrate row operations on page 89 in b2p1}
\footnote{In the book, row operations are done on a matrix determinant. It is altered with the standard matrix style used throughout the book.}

A main problem here is to transform a given matrix $A$ to
what we call a echelon matrix. By an \textcolor{red}{\textit{echelon matrix}} (\underline{echelon form}) is meant a matrix in which the number of zero elements, in each row, preceeding the first non zero element (the \textcolor{red}{\textit{distinguished element}}) increases from row to row until the last row. The last row and some rows preceeding it may consists of zero elements only. This means that:
\footnote{A grammar correction. "This mean that:" is changed into "This means that."}
\begin{align*}
	&a_{ij} = 0   && (j=1,\dots, k) \\
	&a_{i+1j} = 0 && (j=1,\dots, k+1  \text{\space at least})  
\end{align*}
\end{document}