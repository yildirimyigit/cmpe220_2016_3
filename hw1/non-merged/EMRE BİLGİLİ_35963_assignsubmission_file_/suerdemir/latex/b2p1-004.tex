%I used overleaf

\documentclass[11pt]{amsbook}

\usepackage{../HBSuerDemir}	

% ------------------------
\begin{document}
% ++++++++++++++++++++++++++++++++++++++
\hPage{b2p1/004}
% ++++++++++++++++++++++++++++++++++++++

Another definition of a sequence is obtained by giving the first two terms and a relation between $a_n$ and $a_{n-2}$ whose indices differ by 2:

  \begin{exmp}
  Given the sequence defined by 
    \begin{center}
    $a_1 = 3, \ a_2 = 2, \ a_n = \frac{n-1}{n+1}a_{n-2},$
    \end{center}
  a) obtain the first four terms,
  
  \noindent b) find the general term.
  \end{exmp}
  
  \begin{hSolution}
  a) $a_1 = 3, \ a_2 = 2, \ a_3 = \frac{3-1}{3+1}a_1 = \frac{3}{2}, \ a_4 = \frac{3}{5}a_2 = \frac{6}{5}$

  \noindent b) Since indices differ by 2, one evaluates $a_{2n}$ and $a_{2n+1}$ separately. 
  Replacing $n$ by $2n$ in the given relation, one gets  %Book says seperately but dictionary says separately.
    \begin{center}
    $a_{2n} = \frac{2n-1}{2n+1}a_{2n-2}$
    \end{center}
  which, when written for $n  = l, 2, ...$ up to $n$, gives 
    \begin{center}
    $a_4 = \frac{3}{5}a_2$

    $a_6 = \frac{5}{7}a_4$

    $\vdots$ \qquad $\vdots$

    $a_{2n} = \frac{2n-1}{2n+1}a_{2n-2}$
    \end{center}
  which in turn, when multiplied member to member yield 
  \end{hSolution} %Solution continues in next page.

\end{document}