\documentclass[11pt]{amsbook}

\usepackage{../HBSuerDemir}	% ------------------------


\begin{document}

% ++++++++++++++++++++++++++++++++++++++
\hPage{b1p2/246}
% ++++++++++++++++++++++++++++++++++++++

\noindent For $n=4$ we have

\begin{equation*}
  \begin{vmatrix}
   0 & a_1 & a_2 & a_3 \\
    -a_1 & 0 & a_4 & a_5 \\
    -a_3 & -a_4 & 0 & a_6 \\
    -a_3 & -a_5 & -a_6 & 0 \\
  \end{vmatrix}
  = (a_1 a_6 - a_2 a_5 + a_3 a_4)^2
\end{equation*}\\

\par This property is true for all even ordered skew symmetric determinants, but the proof will not be given.
\newline

\begin{enumerate}
    \item [8.] Prove by induction
\end{enumerate}

\begin{equation*}
  P_n(x) = 
  \begin{vmatrix}
   a_0 & a_1 & a_2 & \dots & a_n \\
    -1 & x & 0 & \dots & 0 \\
    0 & -1 & x & & \vdots \\
    \vdots & & & & 0 \\
    0 & \dots & 0 & -1 & x \\
  \end{vmatrix}_{n+1}
  = a_0 x^n + a_1 x^{n-1} + \dots + a_{n-1} x + a_n
\end{equation*}

\begin{proof}
The equality is certainly true for $n=0$. Suppose the property is true for $n=k$, then we should show that the determinant

\begin{equation*}
  P_{k+1} = 
  \begin{vmatrix}
   a_0 & a_1 & a_2 & \dots & a_k & a_{k+1} \\
    -1 & x & 0 & \dots & 0 & 0\\
    0 & & & & & \vdots \\
    \vdots & & & & x & 0\\
    0 & \dots & 0 & & -1 & x \\
  \end{vmatrix}_{k+2}
\end{equation*}\\

\noindent of order $k+2$ is equal to $a_0 x^{k+1} + a_1 x^k + \dots + a_{k+1}$.\newline

\par In the expansion of this determinant with respect to the last column, the cofactor of $x$ is $P_k(x)$:

\begin{equation*}
  P_{k+1}(x) = x P_k(x) + a_{k+1}(-1)^{1+(k+2)}
  \begin{vmatrix}
  -1 & x & 0 & \dots & 0\\
  0\\
  \vdots & & & & 0\\
  & & & & x\\
  0 & \dots & 0 & & -1\\
  \end{vmatrix}_{k+1}
\end{equation*}
\end{proof}

% =======================================================
\end{document}