\documentclass[11pt]{amsbook}

\usepackage{../HBSuerDemir}	% ------------------------


\begin{document}

% ++++++++++++++++++++++++++++++++++++++
\hPage{b2p1/031}
% ++++++++++++++++++++++++++++++++++++++

\par A series $\sum a_n$ such that $\sum \hAbs{a_n}$ is convergent is called a \hDefined{absolutely convergent series}, and the above theorem states that an absolutely convergent series is convergent.
\par As the alternating harmonic series shows, a series may be convergent without being absolutely convergent. Such series are called \hDefined{simply convergent}\footnote{In many textbooks \hDefined{conditional convergent} or \hDefined{semi-convergent} terminologies are used instead of simply convergent} series:

\begin{eqnarray*}
    \sum \hAbs{a_n} \text{(conv.)} &\Longrightarrow& \sum a_n \text{(conv)\quad\dots abs. conv. of}\sum a_n\\
    \sum \hAbs{a_n} \text{(div.)}&\Longrightarrow& \begin{cases} \sum a_n \text{(conv)\quad\dots simply conv. of}\sum a_n\\
    \text{or}\\
    \sum a_n \text{(div)}
    \end{cases}
\end{eqnarray*}

\par There is an essential difference between the absolutely convergent series and simply convergent ones. The absolutely convergent series have the following two properties among others:
\begin{enumerate}
  \item[1.] The terms can be rearranged in any order (rearrangement does not alter the sum).
  \item[2.] Finitely or infinitely many terms may be replaced by their sum.
\end{enumerate}

\par These properties may not be shared by simply convergent series, that is, a rearrangement of terms in a simply convergent series may give a different sum as illustrated by the following example:
\begin{exmp}
    Consider the simply convergent alternating harmonic series
    \[
        S= 1-\frac{1}{2}+\frac{1}{3}-\frac{1}{4}+\dotsb+(-1)^{n+1}\frac{1}{n}+\dotsb
    \]
    Let us rearrange the terms to have the series
    \[
        S'= (1-\frac{1}{2}-\frac{1}{4})+(\frac{1}{3}-\frac{1}{6}-\frac{1}{8})+(\frac{1}{5}-\frac{1}{10}-\frac{1}{12})+\dotsb+(\frac{1}{2n+1}-\frac{1}{4n+2}-\frac{1}{4n+4})+\dotsb
    \]
\end{exmp}


% =======================================================
\end{document}  