\documentclass[11pt, twoside, a4paper]{article}
\usepackage{amsmath}
\usepackage{amssymb}

\linespread{1.6}

\usepackage{natbib}

\usepackage{geometry}
\usepackage{fancyhdr}
\usepackage{amssymb}
\usepackage{amsmath}
\numberwithin{equation}{section}
\numberwithin{figure}{section}
\usepackage{amsfonts}
\usepackage{amsthm}

\geometry{includeheadfoot,
  headheight=14pt,
  left=4cm,
  right=2cm,
  top=1cm,
  bottom=2.5cm}

\usepackage{fancyhdr}
\usepackage[turkish]{babel} 
\usepackage[utf8]{inputenc} 
\usepackage[T1]{fontenc} 

\renewcommand{\headrulewidth}{0pt}
\fancyhf{}
\setcounter{page}{30}
\fancyfoot[LE,RO]{\thepage}
\pagestyle{fancy}

\begin{document}

\textbf{Tanım} \hspace{0.7cm} 1.4.9 \hspace{0.7cm} öb(Ç) olarak gösterilen, çizgedeki öbeklerin sayısına \textit{\underline{çizge öbek sayısı}} \vspace{0.25cm}
\hspace{4.25cm} denir. 

\textbf{Tanım} \hspace{0.7cm} 1.4.10 \hspace{0.4cm} öb($d_i$) olarak gösterilen, $d_i$ düğümünün ilgili bulunduğu \vspace{0.25cm}

\hspace{3.7cm} öbeklerin sayısına,  \textit{\underline{düğüm öbek sayısı}} denir.	\vspace{0.25cm}

Örneğin, Şekil 1.4.2a da verilen çizge için,

\begin{center}
öb(Ç) = $8$
\end{center}

öb($d_0$) = $4$          \hspace{2cm}                   öb($d_1$) = $2$                 \hspace{2cm}                 öb($d_2$) = $2$                \hspace{2cm}                        öb($d_3$) = $2$	  \vspace{0.25cm}

öb($d_4$) = $2$	 \hspace{2cm}	öb($d_5$) = $1$	 \hspace{2cm}	öb($d_6$) = $1$	 \hspace{2cm}	öb($d_7$) = $1$ 	\vspace{0.25cm}

öb($d_8$) = $1$	 \hspace{2cm}	öb($d_9$) = $1$	 \hspace{2cm}	öb($d_10$) = $1$  \hspace{1.85cm}	öb($d_{11}$) = $1$	\vspace{0.25cm}

öb($d_{12}$) = $1$ \vspace{0.25cm}

olduğu görülebilir. Çizgedeki öbek sayıları ile ilgili aşağıdaki teoremi kanıtlamadan \vspace{0.25cm}
\hspace{0.5cm} vereceğiz. \vspace{0.25cm}

\textbf{Teorem} \hspace{0.7cm} 1.4.1 \hspace{0.7cm} Çizgedeki çizge öbek, düğüm öbek ve parça sayıları, 

\begin{center}
$$ ob(C) =  p +  \sum_{(i)} [ob(d_i) - 1] $$
\end{center}

\hspace{4.5cm}  eşitliğini sağlar. \vspace{0.25cm}

Teorem 1.4.1 in doğruluğu, Şekil 1.4.2a daki çizgeden ($p=1$) hemen görülecektir.

\end{document}

