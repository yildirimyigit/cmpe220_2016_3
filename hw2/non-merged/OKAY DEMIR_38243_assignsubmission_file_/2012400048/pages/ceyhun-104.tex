%\documentclass[fleqn]{book}
\documentclass[11pt]{amsbook}

\usepackage[turkish]{babel}

%\usepackage{../HBSuerDemir}	% ------------------------
\usepackage{../Ceyhun}	% ------------------------
\usepackage{../amsTurkish}


\begin{document}
% ++++++++++++++++++++++++++++++++++++++
\hPage{104}
% ++++++++++++++++++++++++++++++++++++++

\reffig{ 2.7.1b } deki çizgenin hem Euler hem de Hamilton olmasına karşın, \reffig{2.7.1a} daki çizge Euler çizgesi değildir. 
\begin{theorem}
	$Ç$ nin Hamilton çizgesi olabilmesi için yeter koşul $A \hPairingCurly{Ç_{+2}}$ nin Hamilton çizgesi olması, gerek koşul ise $A \hPairingCurly{Ç}$ nin Hamilton çizgesi olmasıdır. 
\end{theorem}
\begin{theorem}
	$Ç$ nin Euler çizgesi olabilmesi için gerek ve yeter koşul, $A \hPairingCurly{Ç_{+3}}$ ün Hamilton çizgesi olmasıdır. 
\end{theorem}
\begin{theorem}
	Yol olmayan bağlı $Ç\hPairingParan{d,a}$ çizgesinde her $n\geq d-3$ için, $A^n\hPairingCurly{Ç\hPairingParan{d,a}}$ bir Hamilton çizgesidir.
\end{theorem}
Bu altbölümde, başka anlamıyla, bir çizgenin işlevi olarak tanımlanan çizgeleri inceledik. Daha açık bir deyişle, gerek ayrıt gerekse katkılı ayrıt çizgeleri, bir çizgenin belli özelliklerine ve öğelerine dayanılarak türetilen \emph{işlevsel çizgelerdir.} Bu tür çizgelere başka bir örnek olarak \reffig{2.7.4a} daki $Ç_1$ çizgesini düşünelim. \reffig{2.7.4b}  de $Ç_1$ in olabilecek en çok ayrıtı içeren dolu altçizgeleri gösterilmiştir. Her dolu altçizgeyi bir düğüm olarak düşünürsek, bu düğümlerin,çakışım ilişkisinden $Ç_1$ in örgütü diye 
\end{document}