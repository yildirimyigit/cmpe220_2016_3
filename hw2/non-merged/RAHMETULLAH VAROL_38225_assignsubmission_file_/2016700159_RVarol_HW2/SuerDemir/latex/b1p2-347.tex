\documentclass[11pt]{amsbook}
\usepackage[turkish]{babel}

\usepackage{../HBSuerDemir}	% ------------------------

\usepackage{lipsum}

\setlist[enumerate,1]{%
  label=\arabic*.,
}

\newlist{inlinelist}{enumerate*}{1}
\setlist*[inlinelist,1]{%
  label=(\roman*),
}

\begin{document}
% ++++++++++++++++++++++++++++++++++++++
\hPage{b1p2/347}
% ++++++++++++++++++++++++++++++++++++++
\begin{align*}
	I &= cos^3 x sinx + 3 \int cos^2 x sin^2 x dx \\
	&= cos^3 x sin + 3 \int cos^2 x dx - 3 \int cos^4 x dx \\
	&= cos^3 x sinx + 3 \int cos^2 x dx - 3 \int cos^4 x dx \\
	4I &= cos^3 x sinx + \dfrac{3}{2} \int (1+cos 2x) dx \\
	&= cos^3 x sinx + \dfrac{3}{2} (x+\dfrac{sin 2x}{2})+c_1 \\
	I &= \dfrac{1}{4} cos^3 x sinx + \dfrac{1}{16} sin 2x + \dfrac{3}{8} x + c
\end{align*}

\underline{Properties:} (Indefinite integrals of even, odd functions] \\
Let $e_i(x)$ be even, odd functions respectively. Then recalling properties $De_1(x)=\omega_1(x)$, $D\omega_2(x)=e_2(x)$ (\S 2.1, Exercise 20) we may have the converse. Indeed the following properties hold:

\begin{equation*}
	1. \int e_1dx = \omega_1(x)+c, \qquad 2. \int \omega_2(x) dx = e_2(x)+c
\end{equation*}

\underline{Proof:}
\begin{enumerate}
	\item[1.] Let $F(x)=\int e_1 (x) dx$ without constant of integration. Then
	\begin{equation*}
		F(-x) = \int e_1(x)d(-x) = -\int e_1(-x)dx = -\int e_1 dx = -F(x),
	\end{equation*}
	Showing that $F(x)$ is an odd function, namely $\omega_1(x)$
	
	\item[2.] Proved similarly. \unskip\nobreak\quad\qedsymbol
\end{enumerate}
Discuss periodicity of the integral of a periodic function.

\section*{EXERCISES (5.1)}
\begin{enumerate}
	\item[1.] Simplify the following \\
	\begin{inlinelist}
	\item[a)] $\int d f(x) \qquad$
	\item[b)] $d\int f(x) dx \qquad$
	\item[c)] $\dfrac{d}{dx} \int arccos x dx$
	\end{inlinelist} \\
	\begin{inlinelist}
	\item[d)] $\int \dfrac{d}{df}arccos x dx \qquad$
	\item[e)] $\int d(x^7+x+7)^7 \qquad$
	\item[f)] $\dfrac{d}{dx}\int \dfrac{d}{dx} arcsec x$
	\end{inlinelist}

	\item[2.] If $F_1(x)$, $F_2(x)$ are two primitives of $f(x)$, then show that $c_1 F_1(x)+c_2 F_2(x)$ is a primitive of $f(x)$ when $c_1$, $c_2$
\end{enumerate}


\end{document}