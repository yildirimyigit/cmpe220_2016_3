\documentclass{article}
\usepackage{../HBSuerDemir}
\begin{document}
\hPage{b2p1-128}

Properties: 
  \begin{itemize}
\item[1.] A . B = B .  A ( com. law)
\item[2.] ($\lambda$A). B = A.($\lambda$B) =  $\lambda$(A.B)  
\item[3.] A. ( B+C) = A.B + A.C (dist. law)
\end{itemize}
Proof:
The first two properties are direct consequences of the definition.
To prove the distributive law 
 
\begin {tikzpicture}
\draw[->](0,0) -- (1.33,0);
\draw[->](1.33,0) -- (2.66,0);
\draw[->](2.66,0) -- (4,0);
\draw[->](0,0) -- (2.3,0.6);
\draw[->](0,0) -- (1,1.3);
\draw[->](1,1.3) -- (3.5,2);
\draw[dotted](1,1.3) -- (1.3,0); 
\draw[dotted](3.5,2) -- (2.66,0);
\put(-7, -7) {$O$}
\put(25, 40) {$B$}
\put(70, -10) {$R'$}
\put(65, 17) {$C$}
\put(30, -10) {$B'$}
\put(100, 55) {$R$}
\put(110, 2,2) {$A$}
\end{tikzpicture}
\begin{equation}
A. ( B+C) = A.B + A.C
 \end{equation}


consider $\vec{BR} = \vec{OC}$ (see fig.), and projections B', R' of B, R on OA. Then, 

\begin{itemize}
  \item[] A.(B+C) = A.R
  \item[] =A.R' ( Geom. interp. 1) 
  \item[] =A. (B'+C') 
  \item[] =A.B' + A.C' (collinearity of vectors)
  \item[] =A.B + A.C (Geom. interp. 1)
\end{itemize}

Now we derive the analytic expression

\begin{align}
A.B = a_{1}b{1} + a_{2}b_{2} + a_{3}b_{3} 
for A = ( a_{1},a_{2},a_{3} ), B=( b_{1}, b_{2}, b_{3} ). 
\end{align}
Expanding 
\begin{align}
A.B = (a_{1}i + a_{2}j + a_{3}k). (b_{1}i + b_{2}j + b_{3}k) 
\end{align}

by distributive law, we get nine terms, six of which are zero by properties

\begin {equation}
i.j = 0,  j.k = 0, k.i = 0
\end {equation}

for orthogonal vectors i,j,k and the remaining terms are 
\begin {align}
a_{1}b_{1}, a_{2}b_{2}, a_{3}b_{3} 
\end{align}
by the properties
i.i = 1,  j.j = 1, k.k = 1 for unit vectors i,j,k

\end{document}


