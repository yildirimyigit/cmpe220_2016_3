\documentclass{amsbook}
\usepackage{./Ceyhun}
\usepackage[utf8]{inputenc}
\usepackage{ mathrsfs }

\begin{document}

\subsection*{1.5 UZAKLIK VE OZEK \\} 

Bu alt b�l�mde yaln?z ba?l? �izgeleri de?inece?iz. $d_i$ ve $d_j$ b�yle bir �izgenin iki d�?�m� olsun. �izge ba?l? oldu?u i�in bu d�?�m �ifti aras?nda en az bir yol vard?r. $d_i$ ve $d_j$ d�?�mleri aras?ndaki yollar?n olu?turdu?u, 

${\mathscr{Y}_{i,j}^{1}} = { Y_{i,j}^1 , Y_{i,j}^2 , \dotsc , Y_{i,j}^n}$

yollar y???n?n? d�?�nelim.

\subsubsection*{Tan?m} $\{\mathscr{Y}_{i,j}^{1}\}$ y???n?nda $u(d_i, d_j)$ olarak g�sterilen en k?sa yolun uzunlu?una, $d_i$ d�?�m�n�n $d_j$ d�?�m�ne \underline{uzakl???} denir.


$d_i$ , $d_j$ ve $d_k$ �izgedeki �� d�?�m olsun. Tan?m 1.5.1'den,

\begin{itemize}
    \item [a)] $u(d_i, d_j) >= 0$
    \item [b)] E?er ve ancak $d_i = d_j$ ise, $u(d_i, d_j) = 0$
    \item [c)] $u(d_i, d_j) = u(d_j, d_i)$
    \item [d)] $u(d_i, d_k) + u(d_k, d_j) >= u(d_i, d_j)$ 
\end{itemize}

oldu?u g�r�lebilir. Her d�?�m i�in uzakl???n alabilece?i en b�y�k de?er vard?r.

\end{document}