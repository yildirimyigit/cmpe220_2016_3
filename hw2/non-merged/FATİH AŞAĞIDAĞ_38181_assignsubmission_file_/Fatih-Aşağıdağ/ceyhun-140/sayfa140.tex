\documentclass[11pt]{amsbook}

\usepackage{amsmath}
\usepackage{graphicx}
\graphicspath{ {images/} }
\usepackage[turkish]{babel}

%\usepackage{../HBSuerDemir}	% ------------------------
\usepackage{../Ceyhun}	% ------------------------
\usepackage{../amsTurkish}


\begin{document}
% ++++++++++++++++++++++++++++++++++++++
\hPage{140}
% ++++++++++++++++++++++++++++++++++++++


olarak yazabiliriz. $\overline{B}$' ve $\overline{P}$' sırasıyla, $\overline{B}$ ve $\overline{P}$ 
matrislerinin evriklerini gösterirse, bu örnek için 
\begin{equation*}
\overline{P}\overline{B}' = 0 \text{ ya da } \overline{B}\overline{P}' = 0
\end{equation*}
olduğu hemen görülecektir. Bu özelliğin bütün  
çizgeler için de doğru olduğunu göstermek için, 
P nin i ninci dizeği $p_i$ ve $\overline{B}$ nin j ninci dizeği 
$b_j$ yi düşiinelim. Eğer $d_i$ düğümü, $Ç_j$ çevresinde 
ise $p_i$ de ve $b_j$ de ortak olan yalnız iki tane bire 
eşit terim vardır ve bu dizekler iki tabanına göre 
çarpıldıklarında sıfır vereceklerdir. Eğer$d_i$
düğümü $Ç_j$ çevresinde değilse, $p_i$ ve $b_j$ de bire 
eşit ortak terim yoktur ve bu dizekıerin çarpımı 
yine sıfır verecektir.

Tanım 3.2.6 dan t-çevrelerin bağımsız olduğu 
görülmektedir. Öyleyse çevre matrisinin, belli 
bir ağaca göre tanımlanan t-çevrelerden oluşan 
altmatrisi herzaman, 
\begin{equation*}
B_t =
  \begin{bmatrix}

B_1 & I

  \end{bmatrix}
\end{equation*}
biçiminde yazılabilir. $B_t$ ye $\textit{t-çevre matrisi}$
diyeceğiz. $B_1$ ve I nın dikeçleri sırasıyla,
dallara ve krişlere karşıdüşmektedir. Örneğin,
Şekil 3.3.1 de kalın çizgilerle belirtilen ağaca ilişkin t-çevre matrisi,
\end{document}