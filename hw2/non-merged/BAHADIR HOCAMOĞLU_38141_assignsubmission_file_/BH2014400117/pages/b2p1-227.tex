\documentclass{amsbook}
\usepackage{../HBSuerDemir}
   
\begin{document}
    \hPage{b2p1/227}
    \pagenumbering{gobble}
    
    \section*{B. SPACE CURVES}
        \subsection*{Definitions:}
        \paragraph
        A vector function
        \begin{equation*}
            r(t) = x(t)i + y(t)j + z(t)k,  \ t_\varepsilon [\alpha ,\beta]
        \end{equation*}
        
        considered as a variable position vector $r(t)\ =\ P(t)$ defines a curve $\tau$(or an arc of curve) as the locus of $P(x(t),\ y(t),\ z(t))$ when $t$ varies in the interval $[\alpha,\ \beta]$, of which the parametric equations are
      
        \[\tau : \left\{
        \begin{array}{ll}
        x = x(t) & \\
        y = y(t), & t_\varepsilon [\alpha ,\beta] \\
        z = z(t)  & \\
        \end{array} 
        \right. \]
      
      The initial point of $\tau$ is $A(t = \alpha)$ and the end point is $B(t = \beta)$. $\tau$ is said to be \underline{oriented} from $A$ to $B$ and the sense is indicated by an érrow put on the curve.
      
      \paragraph
      \tau is a \underline{plane curve} or a \underline{skew curve} according as it lies or does not lie on the plane. A plane curve or skew curve is called a \underline{space curve}.
      
      \paragraph
      Elimination of $t$ between two coordinates gives a relation between these coordinates showing that $\tau$ lies on a cylinder. Such a cylinder is called a \underline{projecting cylinder}. $\tau$ defines three projecting cylinders.
      
      \paragraph
      The space curve $\tau$ is \underline{closed} if  $r(\alpha) = r(\beta)$ (or if  $A \cong  B$), and called \underline{simple} if $\tau$ does not intersect itself. $\tau$ is said to be \underline{smooth} if admits tangent line
      of every point of it.
      
      \paragraph
      Lines are simple examples of space curves. Below we give an interesting curve which is skew.

\end{document}