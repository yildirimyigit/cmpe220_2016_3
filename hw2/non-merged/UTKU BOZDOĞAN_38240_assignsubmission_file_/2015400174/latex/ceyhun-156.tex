\documentclass[12pt]{article}

\usepackage{../Ceyhun}
\usepackage{../amsTurkish}

\begin{document}

	\hPage{ceyhun-156}

	\underline{3.4 t-Kesitleme matrisinin gerşekleştirimi} \\

	elde edilir. \\

	Burada $\circleddash$ , Tanım 3.4.2 de açıklandığı gibi özel bir matris çıkışma işlemini göstermektedir. Tanım 3.4.2 den, {\LARGE M}$(\text{i})_{1}$ ve 	{\LARGE M}$(\text{i})_{2}$ nin sırasıyla, $\overline{{\LARGE Ç}}_{1}$ ve $\overline{{\LARGE Ç}}_{2}$ çizgelerinin t-kesitleme matrisleri olduğu hemen görülecektir. Ayrıca, {\LARGE M}(i) matrisindeki i kesitlemesi bir \textit{çakışım kümesi} dir de. \\

	{\LARGE M}$(\text{i})_{1}$ matrisi, {\LARGE Ç} ye ilişkin bir altçizgenin t-kesitleme matrisi olduğuna göre, bu kez {\LARGE H}(j) (j $\neq$ i) matrisi ve bu matrise ilişkin {\LARGE M}$(ij)_{1}$ ve  {\LARGE M}$(ij)_{2}$ matrislerini elde edebiliriz. Bu matrislerin temel özelliği, her birinde i ve j kesitlemelerinin birer çakışım kümesi olmasıdır. Bu işlemi yeterince yenilersek, bütün dizekleri birer çakışım kümesi olan \emph{temel} {\LARGE M}-\emph{matrislerini} buluruz. Her bir temel {\LARGE M}-matrisi çakışım matrisi olduğu için, ilişkin altçizgelerin bulunması bir sorun değildir. Bu altçizgelere Şekil 3.4.4b de açıklanan işlemin tersinin uygulanması, ${\LARGE Q}_{t}$ matrisine ilişkin aranan çizgeyi verecektir. Eğer elde edilen temel {\LARGE M}-matrislerinden herhangi biri çakışım matrisinin özelliklerini sağlamıyorsa, ilişkin t-kesitleme matrisi de gerçeklemez demektir. Ancak, burada bir saptama yapmamız uygun olur. {\LARGE H}(i) matrisinin parçalanması doğru yapılmamışsa yanlış {\LARGE M}-matrislerinin elde edilmesi doğaldır. Demek ki, {\LARGE H}(i) matrisinin bütün değişik

\end{document}