\documentclass{article}
\usepackage{HBMath}
\usepackage{Feyzioglu}

\begin{document}
   
   \hspace{10mm}
   $A = \bigcup_{\substack{ a \in A}}
   \hPairingBraket{a}$ 
   and if 
   $\hPairingBraket{a} \neq \hPairingBraket{b}$, 
   then 
   $\hPairingBraket{a} \hIntersection \hPairingBraket{b} = \emptyset$. \\
   
   Conversely, let \\
   
  
   \hspace{10mm} $A = \bigcup_{
   \substack{ a \in A}
   } 
   P_{i}$,
   $P_{i} \hIntersection P_{j} = \emptyset$ 
   if
   $i\neq j$ \\
   
   be a union of nonempty, mutually disjoint sets $P_{i}$, indexed by $I$. Then there is an equivalence relation on $A$ such that the
   $P_{i}$'s are the equivalence classes under this relation.
   
   \begin{proof}
   
    First we prove 
    $A = \bigcup_{
    \substack{ a \in A}
    } \hPairingBraket{a}$. 
    For any 
    $a \in A$,
    we have 
    $\hPairingBraket{a} \subseteq A$,
    hence 
    $\bigcup_{
    \substack{ a \in A}
    } 
    \hPairingBraket{a} \subseteq A$. 
    Also, if 
    $a \in A$, 
    then
    $a \in \hPairingBraket{a}$ by reflexivity, so 
    $a \in \bigcup_{
    \substack{ a \in A}
    } 
    \hPairingBraket{a} $ 
    and
    $A \subseteq \bigcup_{
    \substack{ a \in A}
    } 
    \hPairingBraket{a}$. 
    So
    $A = \bigcup_{
    \substack{ a \in A}
    } 
    \hPairingBraket{a}$.
   
   \end{proof}
   
   Now we must prove that distinct equivalence classes are disjoint. We prove its contrapositive, which is logically the same: if two equivalence classes are not disjoint, then they are identical. Suppose that the equivalence classes $\hPairingBraket{a}$ and $\hPairingBraket{b}$ are not disjoint. This means there is a $c$ in $A$ such that $c \in \hPairingBraket{a}$ and $c \in \hPairingBraket{b}$. Hence\\
   
   \hspace{10mm}
   $c\sim a$ and $c\sim b $ 
   
   \hspace{10mm} 
   $a\sim c$ and $c\sim b$  \hspace{11mm}(by symmetry)
   
   (1)\hspace{6mm}  
   $a\sim b$ \hspace{25mm} (by transitivity)
   
   (2)\hspace{6mm}  
   $b\sim a$  \hspace{25mm} (by symmetry).\\
   
   We want to prove 
   $\hPairingBraket{a} = \hPairingBraket{b}$.
   To this end, we have to prove
   $\hPairingBraket{a} \subseteq \hPairingBraket{b}$ 
   and also 
   $\hPairingBraket{b} \subseteq \hPairingBraket{a}$. 
   Let us prove 
   $\hPairingBraket{a} \subseteq \hPairingBraket{b}$. 
   If 
   $x \in \hPairingBraket{a}$, 
   then 
   $x \sim a$ and $a \sim b$ 
   by (1), then 
   $x \sim b$
   by transitivity, then 
   $x \in \hPairingBraket{b}$,
   so 
   $\hPairingBraket{a} \subseteq \hPairingBraket{b}$. 
   Similarly, if 
   $y \in \hPairingBraket{b}$, then $y \sim b$, 
   then
   $y \sim b$ 
   and
   $b \sim a$ 
   by (2), then 
   $y \sim a$ 
   by transitivity, then 
   $y \in \hPairingBraket{a}$, 
   so 
   $\hPairingBraket{b} \subseteq \hPairingBraket{a}$. 
   Hence $\hPairingBraket{a} = \hPairingBraket{b}$ 
   if 
   $\hPairingBraket{a}$ 
   and 
   $\hPairingBraket{b}$ 
   are not disjoint. This completes the proof of the first assertion.\\
   \\
   Now the converse. Let  
   $A = \bigcup_{
   \substack{ a \in A}
   } P_{i}$, 
   where any two distinct $P_{i}$'s are disjoint. We want to define an equivalence relation on $A$ and want the $P_{i}$'s to be the equivalence classes. How do we accomplish this? Well, if the $P_{i}$ are to be the equivalence classes, we had better call two elements equivalent if they belong to  one and the same $P_{i_{0}}$.\\
   \\
   Let 
   $a \in A$. 
   Since 
   $A = \bigcup_{
   \substack{ a \in A}
   }
   P_{i}$, 
   we see that 
   $a \in P_{i_{0}}$,
   for some 
   $i_{0} \in I$.
   This index $i_{0}$ is uniquely determined by $a$. That is to say, $a$ cannot belong to
   
   

\end{document}
