\documentclass[11pt]{amsbook}
\usepackage[turkish]{babel}

\usepackage{lipsum}
\usepackage{amsmath}
\usepackage{enumitem}
\usepackage{../HBSuerDemir}

\begin{document}
% ++++++++++++++++++++++++++++++++++++++
\hPage{feyzioglu-50}
% ++++++++++++++++++++++++++++++++++++++
In addition one proves that there are integers $x_1,x_2,...,x_{n-1},x_n$ such that
\begin{equation*}
	a_1 x_1 + a_2 x_2 + ... + a_{n-1} x_{n-1} + a_n x_n = (a_1,a_2,....,a_{n-1},a_n)
\end{equation*}

If $(a_1,a_2,....,a_{n-1},a_n)=1$  we say that $a_1,a_2,....,a_{n-1},a_n$ are relatively prime.
In this case, there are integers $x_1,x_2,...,x_{n-1},x_n$ satisfying

\begin{equation*}
	a_1 x_1 + a_2 x_2 + ... + a_{n-1} x_{n-1} + a_n x_n = 1
\end{equation*}

The proofs of these assertions are left to the reader.\\

Our final topic in this paragraph will be the least common multiple of
two nonzero integers. If $a,b \in \mathbb{Z}$ and $a|b$, we say that $b$ is a multiple of $a$.\\

\textbf{5.19 Theorem:} Let $a,b \in Z$, neither of them zero (i.e., $a \neq 0 \neq b$). Then
there is a unique integer such that\

\begin{enumerate}[label=(\alph*)]
\item $a|m$ and $b|m$,
\item for all $m_1 \in \mathbb{Z}$, if $a|m_1$ and $b|m_1$, then $m|m_1$,
\item $m > 0$
\end{enumerate}\

\textbf{Proof} The proof will be similar to that of Theorem 5.4. We consider the set $V = \{ n \in \mathbb{N} : a|n \text{ and } b|n \}$. This set is not empty, since, for example $|ab|$ is in $V$ (here we use the hypothesis $a \neq 0 \neq b$). We choose the smallest positive integer in $V$. Let it be called $m$. Thus $m > 0$ and $m$ satisfies (iii). Also, $a|m$ and $b|m$ since $m \in V$, and $m$ satisfies (i). It remains to show that $m$ satisfies (ii). 

Suppose $m_1 \in \mathbb{Z}$, and $a|m_1$ and $b|m_1$. We divide $m_1$ by $m$ and get, say, $m_1 = qm + r$, where $q,r \in \mathbb{Z}$ and $0 \leq r < m$. Since $a|m$, $a|m_1$ and $b|m$, $b|m_l$, the equation $m_1 = qm + r$ yields that, $a|r$ and $b|r$. Hence $r \in V$. We know $0 \leq r < m$. If $r$ were not zero, then $r$ would be a natural number in $V$ smaller than the smallest natural number $m$ in $V$, which is absurd. Thus $r = 0$, so $m_1 = qm$, and $m|m1$. This shows that in satisfies (ii).

Now the uniqueness of $m$. Suppose $m'$ satisfies the conditions (i), (ii), (iii), too. Then $a|m'$, $b|m'$ by (i), and so $m'|m$' by (ii). Also, $a|m$, $b|m$ by (i), and so $m'|m$ by (ii). Hence $m|m'$ and $m'|m$. By Lemma 5.2(12), we obtain 


\end{document}