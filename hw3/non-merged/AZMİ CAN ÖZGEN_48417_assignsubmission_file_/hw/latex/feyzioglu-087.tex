\documentclass[11pt]{amsbook}
\usepackage{amsmath}
\usepackage{amsfonts}
\usepackage{amsthm}
\usepackage{../HBSuerDemir}
% \newtheorem*{lemma}{Lemma}

\title{feyzioglu_087}
\author{feyzioglu}

\begin{document}
% -------------
\hPage{feyzioglu-087}
% -------------

\begin{align*}
	&= b((a_{1}a_{2} \ldots a_{m - 1})a_m)  \\
	&= b(a_{1}a_{2} \ldots a_{m - 1}a_m)
\end{align*}
as was to be proved. \hfill\(\Box\) \\[0.5in]
Lemma 8.11 gives a new proof of Lemma 8.10 when we choose \\
$ a_1 = a_2 = \ldots = a_m = a $ and replace $ b $ by $ b^n $.\\[0.1in]
We proved in Lemma 8.3 that the product of $ n $ elements in a group
(or in a set an associative multiplication on it) is independent
of the mode of putting parantheses. When the elements commute, the
product is also independent of the order of elements. \\[0.4in]
\textbf{8.12 Lemma:} \textit{Let G be a nonempty set with an associative
multiplication on it. For all $ n \in \mathbb N $,
for all $ a_1, a_2, \ldots, a_n \in G $ such that}

\begin{center}
	$ a_{i}a_{j} = a_{j}a_{i} $ whenever $ i,j = 1,2,\ldots,n $
\end{center}
\textit{there holds}
\begin{center}
	$ a_{k_{1}}a_{k_{2}} \ldots a_{k_{n}} = a_{1}a_{2} \ldots a_{n} $
\end{center}
\textit{for each arrangement $ k_1, k_2, \ldots, k_n $ of $ 1,2, \ldots, n $
(i.e., for each $ k_1, k_2, \ldots, k_n $ such that}
$ \{k_1, k_2, \ldots, k_n\} = \{1,2,\ldots,n\} $).\\[0.1in]
\textbf{Proof:} By induction on $ n $. The case $ n = 1 $ is trivial.
Now assume $ n \geq 2 $ and the claim is proved for $ n - 1 $, for all
pairwise commuting elements $ b_1, b_2, \ldots, b_{n - 1} $ of $ G $,
for all arrangements of $ 1,2, \ldots, n - 1 $. Let $ a_1, a_2, \ldots, a_n $
be $ n $ arbitrary pairwise commuting elements of $ G $ and let
$ k_1, k_2, \ldots, k_n $ be an arbitrary arrangement of
$ 1,2, \ldots, n $. Then $ n = k_j $ for some $ j \in \{1,2, \ldots, n \} $.
We have
\begin{align*}
	a_{k_1}a_{k_2} \ldots a_{k_n} &= (a_{k_1} \ldots a_{k_{j-1}})a_{k_j}(a_{k_{j+1}} \ldots a_{k_n})   \\
								  &= (a_{k_1} \ldots a_{k_{j-1}})(a_{k_j}(a_{k_{j+1}} \ldots a_{k_n})) \\
								  &= (a_{k_1} \ldots a_{k_{j-1}})(({a_{k_{j+1}} \ldots a_{k_n}})a_{k_j})  \\
								  &= ((a_{k_1} \ldots a_{k_{j-1}})({a_{k_{j+1}} \ldots a_{k_n}}))a_{k_j}  \\
								  &= (a_{k_1} \ldots a_{k_{j-1}}{a_{k_{j+1}} \ldots a_{k_n}})a_n
\end{align*}
and here $ k_1, \ldots, k_{j - 1}, k_{j + 1}, \ldots, k_n $
are simply the numbers $ 1,2, \ldots, n - 1 $ in some order.
By the inductive hypothesis, applied to the
elements $ a_1, a_2, \ldots, a_{n - 1} $

\end{document}
