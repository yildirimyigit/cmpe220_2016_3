\documentclass[11pt]{amsbook}

\usepackage{../HBSuerDemir}	% ------------------------

\begin{document}

% ++++++++++++++++++++++++++++++++++++++
\hPage{feyzioglu/93}
% ++++++++++++++++++++++++++++++++++++++

\begin{enumerate}
	\item[(ii)]
		for all $a \in H$, we have $a^{-1} \in H$ ($H$ is closed under the forming of inverses).
\end{enumerate}\par

So we can dispense with checking $1 \in H$ when we know $H \neq \emptyset$. On the other hand, when we do not know a priori that $H \neq \emptyset$, the easiest way to ascertain $H \neq \emptyset$ may be to check that $1 \in H$.\par
When our subset is finite, we can do even better.\par

\begin{lem}
\begin{enumerate}
	\item
		Let G be a group and let H be a nonempty finite subset of G. Then H is a subgroup of G if and only if H is closed under multiplication.
	\item
		Let G be a finite group and let H be a nonempty subset of G. Then H, is a subgroup of G if and only if H is closed under multiplication.
\end{enumerate}
\end{lem}

\begin{proof}
\begin{enumerate}
	\item \label{1}
		We prove that 9.2(ii) follows from 9.2(i)  when $H$ is finite, so that 9.2(i) and 9.2(ii) are together equivalent to 9.2(i), which is the claim. So, for all $a \in H$, we must show that $a^{-1} \in H$ under the assumption that $H$ is finite and closed under multiplication.\par
If $a \in H$ and $H$ is closed under multiplication, we have $aa = a^{2} \in H$, $a^{2} a = a^{3} \in H, \ldots$, in general $a^{n} \in H$ for all $n \in \mathbb{N}$. The infinitely many elements $a,a^{2},a^{3},\ldots ,a^{n} , \ldots of H$ cannot be all distinct, because H is a finite set. Thus $a^{m}= a^{k}$ for some $m,k \in \mathbb{N}, m \neq k$. Without loss of generality, let us assume
$m > k$. Then
\begin{align*}
	a^{m-k-1} a = a^{m-k} = a^{m}a^{-k} = a^{m}(a^{k})^{-1} = a^{m}(a^{m})^{-1} = 1.
\end{align*}
so that $a^{-1} = a^{m-k-1} \in H$. So $H$ is closed under the forming of inverses.

	\item
		This follows from \ref{1}, since any subset of a finite set is finite.
\end{enumerate}
\end{proof}

\begin{exmp}
\begin{enumerate}[label=(a)]
	\item
		For any group G, the subsets \{1\} and G are subgroups of G. Here \{1\} is called the trivial subgroup of G.
	\item
		If $K \leq H$ and $H \leq G$, the K is clearly a subgroup of G.
\end{enumerate}
\end{exmp}


\end{document} 