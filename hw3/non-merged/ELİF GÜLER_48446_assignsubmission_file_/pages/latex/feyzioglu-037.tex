\documentclass[11pt]{amsbook}

\usepackage{../HBSuerDemir}	% ------------------------


\begin{document}

% ++++++++++++++++++++++++++++++++++++++
\hPage{feyzioglu-037}
% ++++++++++++++++++++++++++++++++++++++


% =====================================
\begin{thm} \footnote{The theorem starts in the previous page. Begin tag should be removed}
	(The integer $q$ is called the \hDefined{quotient}, and $r$ is called 
	the \hDefined{remainder} obtained when $a$ is divided by $b$.)
\end{thm}

% =====================================
\begin{proof}
	There are two claims in this theorem: (1) that there are integers
	$q,r$, with the stated properties and (2) that these are unique, that is, 
	the pair of integers $q,r$ is the only one which has the stated properties.
	The proof of this theorem will accordingly consist of two parts. In the first
	part, we prove the existence of $q,r$, in the second part, their uniqueness.

% =====================================
	Existence. Consider the set $T =$ \{$a - ub : u \in \mathbb{Z}$\} $\subseteq \mathbb{Z}$
	This set $T$ conains nonnegative integers (for example, $a-(-|a|)b$ is nonnegative).
	We choose the smallest nonnegative integer in $T$. Let it be called $r$. Thus $r \geqslant 0$
	and, by the very definition of $T$, we infer $r = a - qb$ for some $q \in \mathbb{Z}$. 
	We claim $r < b$. If we had $r \geqslant b$, then, since $b>0$, we would get

	\[ r > r-b = a - (q+1)b \geqslant 0\]
	and $r-b$ would be a nonnegative integer in $T$, smaller than the smallest nonnegative integer 
	in $T$, which is absurd. So $r \geqslant b$ is impossible and $r<b$. Hence there are integers
	$q,r$ such that

	\[ a = qb + r, \qquad 0 \leqslant r < b\]

% =====================================
	Uniqueness. Let $a = qb + r$, $0 \leqslant r < b$, and $a = q_1b + r_1$, $0 \leqslant r_1 < b$, 
	where $q,r,q_1,r_1$ are integers. We wish to prove $q_1 = q$ and $r_1 = r$. It suffices to prove
	$q_1 = q$, for then we would get $r_1 = a - q_1b = a - qb = r$ also. Suppose, by way of 
	contradiction that $q \neq q_1$. Then there are two possibilities: $q > q_1$ or $q < q_1$. 
	Interchanging $q,r$ with $q_1,r_1$ if necessary, we may assume $q > q_1$ without loss of 
	generality (make sure that you understand this reasoning). From $q>q_1$, we get 
	$q - q_1 \geqslant 1$, hence

	\[
	 r_1 = r_1 - 0 \geqslant
					r_1 - r = (a - q_1b) - (a - qb) = 
										(q - q_1)b \geqslant 1b = b
	\]
	a contradiction. So $q_1 = q$ and $r_1 = r$.

\end{proof}

% =====================================

This theorem formalizes what everybody learns at primary school: when
we divide $a$ by $b$, we get a quotient, and a remainder smaller than $b$. At
primary school, one learns it in the case $a$ is positive, but here $a$ can be
negative. Also, division is carried out by successive subtractions

\[
  \begin{array}{r|r}
     a \phantom{11} & b \\ \cline{2-2}
       ... \phantom{1} & q\\
         ...\\
           r 
  \end{array}
\]

\end{document}