\documentclass{article}
\usepackage[utf8]{inputenc}
\usepackage{HBSuerDemir}
\usepackage{Ceyhun}
\newtheorem{theorem}{Theorem}[section]
\newtheorem{corollary}{Corollary}[theorem]
\newtheorem{lemma}[theorem]{Lemma}

\begin{document}
\hpage{vol1-59}

and b). 2) Evaluate $F(a)$ and $F(b)$. 3) Put $I(j) = F(b)- F(a)$. There are
many functions F with $F'(x) = f(x)$ for $all x \ni [a,b]$. For two different
choices $F_1$ and $F_2$, we have $F_1(b) \neq  F_2(b)$ and$ F_1(a) \neq F_2(a)$ in general. So we may suspect that $F_1(b) - F_1(a) \neq F_2(b) - F_2(a)$.In order to show that I is a well defined function, we must prove $F_1 (b) - F_1 (a) = F_2(b) - F_2(a)$
whenever $F_1$ and $F_2$ are functions on [a,b] such that $F_1'(x) = f(x) = F_2'(x)$
for all $x \ni[a,b]$. We know from the calculus that, when $F_1 $and$ F_2$ have
this property, there is a constant c such that $F_1(x) = F_2(x) + c$ for all
$x \ni [a,b]$. So $F_1(b) - F_1(a) = (F_2(b) +c) - (F_2(a) + c)= F_2(b )- F_2(a)$. Therefore,I is well defined. 

After this lengthy digression, we return to the integers mod n and to the
"operations" $\oplus$ and $\otimes$.


\begin{lemma}
$\oplus and \otimes are well defined operations on \mathbb{Z}_n$
\end{lemma}

\textbf{Proof} we are to prove $\overline{a} \oplus \overline{b} = \overline{a'} \oplus \overline{b'}$ and $\overline{a} \otimes \overline{b} = \overline{a'} \otimes \overline{b'}$ whenever $\overline{a} = \overline{a'}$ and $\overline{b} = \overline{b'}$ in $\mathbb{Z}_n$ (different names for identical residue classes should not yield different results). This follows from Lemma 6.1 Indeed, if $\overline{a} = \overline{a'}$ and $\overline{b} = \overline{b'}$, then $a = a' (mod  n)$ and $b = b' (mod  n)$ by definition, so we obtain $a + b = a' + b' (mod  n)$ and $\overline{ab} = \overline{a'b'} (mod  n)$ hence $\overline{a+b} = \overline{a' + b'}$ and $\overline{ab} = \overline{a'b'}$ by Lemma 6.1, hence $\overline{a+b} = \overline{a'+b'}$ and $\overline{ab} = \overline{a'b'}$, which gives $\overline{a} \oplus \overline{b} = \overline{a+b} = \overline{a'+b'} = \overline{a'} \oplus \overline{b'} $ and $\overline{a} \otimes \overline{b} = \overline{ab} = \overline{a'b'} = \overline{a'} \otimes \overline{b'}$.

Having proved that $\oplus$ and $\otimes$ are well defined operations on $ \mathbb{Z}_n$, we proceed to show that $\oplus$ and $\otimes$ many (but not all) properties of the usual addition and multiplication of integers. First we simplify our notation. From now on, we write + and · instead of $\oplus$ and $\otimes$. In fact, we shell even drop · and use simply juxtaposition to denote a product of two integers mod n. Thus we will have $\overline{a} + \overline{b} = \overline{a+b} and \overline{a} · \overline{b} = \overline{ab}$ or simply $\overline{a}\overline{b} = \overline{ab}$. The reader should note that the same sign "+" is used to donate two very distinct operations: $\oplus$ in the odd notation and the usual addition of integers. If anything, they are defined on distinct sets $\mathbb{Z}_n$ and $\mathbb{Z}$. The same remarks apply to multiplication.
\end{document}
