\documentclass{amsbook}
\usepackage{../HBSuerDemir}
\author{Murat Buldu}

\begin{document}
\hPage{71}
\[  a \circ 2 = a2 - a - 2 + 2 = 2a - a = a \]
for\footnote{ Correction: 'far' in the original and corrected with 'for'. } all \( a \in \mathbb{Q} \setminus \{1\} \). Since \(2 \in \mathbb{Q} \setminus \{1\} \), 2 is indeed a right identity in \( \mathbb{Q} \setminus \{1\} \).
\begin{hEnumerateAlpha}
    \item[(iv)]
        For all \( a \in \mathbb{Q} \setminus \{1\} \), we must find an \( x \in \mathbb{Q} \setminus \{1\} \) such that \( a \circ x = 2 \). Well, this gives 
        \begin{align*}
            ax - a - x + 2 &= 2 \\
            ax - a - x + 1 &= 1 \\
            (a - 1)(x - 1) &= 1 \\
            x - 1 &= 1/(a - 1) \\
            x &= a/(a - 1),
        \end{align*}
        which is meaningful since \( a \neq 1 \). We have not proved yet that \( a/(a-1) \) is a right inverse of a. We showed only that a right inverse of \( a \in \mathbb{Q} \setminus \{1\} \), if it exists at all, has to be \( a/(a-1) \). We must now show that \( a \circ a/(a-1) = 2 \) for all \( a \in \mathbb{Q} \setminus \{1\} \) and also that \( a/(a - 1) \in \mathbb{Q} \setminus \{1\} \). Good. We have 
        \begin{align*}
            a \circ a/(a-1) &= a(a/(a-1)) - a - (a/(a-1)) + 2 \\
            &= (a - 1)(a/(a-1)) - a + 2 \\
            &= 2, 
        \end{align*}
        and also \( a/(a - 1) \neq 1 \) for \( a/(a - 1) \in \mathbb{Q} \) and \( a/(a - 1) = 1 \) would imply that \( a = a - 1 \), hence \( 0 = 1 \), which is absurd. \\
        Since all the group axioms hold, \( \mathbb{Q} \setminus \{1\} \) is a group under \( \circ  \).\\
\end{hEnumerateAlpha}
\begin{hEnumerateAlpha}
    \item[(b)]
        Let us define an operation * on \( \mathbb{Z} \) by putting \( a * b = a + b + 2 \) for all \( a,b \in \mathbb{Z} \). Does \( \mathbb{Z} \) form a group under *?
        \begin{hEnumerateAlpha}
            \item[(i)]
                For any \( a,b \in \mathbb{Z} \), \( a * b = a + b + 2 \) is an integer. So \( \mathbb{Z} \) is closed under *.
            \item[(ii)]
                For all \( a,b,c \in \mathbb{Z} \), we ask if \( (a * b) * c = a * (b * c) \). We have
                \begin{align*}
                    (a * b) * c &= (a + b + 2) * c \\
                    &= (a + b + 2) + c + 2 \\
                    &= a + (b + 2 + c) + 2 \\
                    &= a + (b + c + 2) + 2 \\
                    &= a + (b * c) + 2 \\
                    &= a * (b * c).
                \end{align*}
                So * is associative.
            \item[(iii)]
                Is there an integer \( e \in \mathbb{Z} \) such that \( a * e = a \) for all \( a \in \mathbb{Z} \)? Well, this gives \( a + e + 2 = a \) and \( e = -2 \). Let us check whether \( -2 \) is really a right identity element. We observe that \( a * -2 = a + (-2) + 2 = a \) for all \( a \in \mathbb{Z} \). So \( -2 \) is a right identity element.
            \item[(iv)]
                Does each integer \( a \) have a right inverse in \( \mathbb{Z} \)? The condition \( a * x = -2 \) yields
                \begin{align*}
                    a + x + 2 &= -2 
                \end{align*}
        \end{hEnumerateAlpha}
\end{hEnumerateAlpha}

\end{document}