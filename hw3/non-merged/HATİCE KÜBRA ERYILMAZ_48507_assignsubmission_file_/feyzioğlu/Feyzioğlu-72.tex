\documentclass[11pt]{article}
\usepackage[utf8]{inputenc}
\usepackage{HBSuerDemir}

\title{feyzioğlu-072}
\author{hkubraeryilmaz }
\date{December 2016}

\begin{document}
\hPage{feyzioğlu-072}

\begin{align*}
    x &= -4-a \in \mathbb{Z} 
\end{align*}
$-4-a$ is needed a right inverse of a since $a * (-4-a) = a + (-4-a) + 2 = -2$. Therefore $\mathbb{Z}$ is a group with respect to $*$.
(C) Let A be a nonempty set and let \& be the set of all subsets of A. The elements of \& are thus subsets of A. Consider the forming of symmetric differences ($\S$ 1, Ex.7). \& is a group under $\Delta$ :
\footnote{Actually the sign is not \& but mirror of it, but I couldnt find it in web so a used this instead.}

\begin{hEnumerateRoman}
    \item For all S,T $\in$ \& , S $\Delta$ T is a subset of A, so S, T $\in$ \& and \& is closed under $\Delta$.
    \item $\Delta$ is associative ($\S$ 1, Ex.8).
    \item $\O$ is a right identity ($\S$ 1, Ex.8).
    \item Each element S of \& has a right inverse, namely S itself, as S $\Delta$ S = $\O$  for all S $\in$ \& ($\S$ 1, Ex.8). 
    
\end{hEnumerateRoman} $\par$

So \& is a group under $\Delta$.

We have seen many examples of groups (G,$\circ$), the underlying set G is infinite, in some finite. The number of elements of G, more precisely the cardinality of G, is called the \textit{order} of the group (G,$\circ$). We denote the order of (G,$\circ$) by $|G|$. A group (G,$\circ$) is called a \textit{finite} group if $|G|$ is finite, and an \textit{infinite} group if $|G|$ is infinite. One might distinguish between various infinite cardinalities, but we will not do so in this book. When the order of a group (G,$\circ$) is infinite, we write $|G| = \infty$. The symbol $\infty$ will stand for all types of infinities.

A. Cayley (1821-1895) introduced a convenient device for investigating groups. Let (G,$\circ$) be a finite group. We make a table that displays a $\circ$ b for each a, b $\in$ G. We divide a square into $|G|^2$ parts by dividing the sides into $|G|$ parts. Each one of the rows will be indexed by an element of the group, usually written in the left of the row. Likewise, each one of the columns will be indexed by an element of the group, usually written above the column. Each element will index only one row and only one column. It is customary to use the same ordering of the elements to index the rows and columns. Also, the first row and the first column are customarily indexed by the identity element of the group. In the cell where the row of $a \in G \ and \ b \in G$ meet, we write down a $\circ$ b. This square is known as the \textit{Cayley table} or the \textit{operation table (multiplication or addition table}, as the case may be) \textit{of the group (G,$\circ$).}      

\end{document}