\documentclass[11pt]{amsbook}
\usepackage[turkish]{babel}
\usepackage{../amsTurkish}
\usepackage{../Ceyhun}
\begin{document}
\hPage{ceyhun-17}
Şekil 1.2.1 deki Ç\hPairingParan{9,14} çizgesine göre, 
	\begin{flalign*}
	G_{1,5}=\hPairingParan{a_{1},a_{7},a_{11},a_{10},a_{8},a_{6},a_{4},a_{3},a_{5}}
	\end{flalign*}
uç düğümleri $d_{1}$ ve $d_{5}$, uzunluğu ise $9$ olan bir gezidir.
	\begin{definition}
		Bir düğüme çakışık olan ayrıtların sayısına, \underline{düğümlerin kertesi} denir.
	\end{definition}
	$d_{i}$ düğümünün kertesi $k_{i}$ ile gösterelim. Her ayrıt yalnız iki düğüme çakışık olabileceğinden,
	\begin{flalign*}
	&2a=\sum_{i=1}^{d}k_{i}&&
	\end{flalign*}
	eşitliğini hemen yazabiliriz.
	\begin{theorem}
		Bir çizgede, kertesi teksayı olan düğümlerin sayısı çiftsayıdır.
	\end{theorem}
	Tanıt: \\
	Kertesi $i$ olan düğümlerin sayısını $t_{i}$ ile gösterirsek,
	\begin{flalign*}
		2a&=\sum_{\hPairingParan{i}}k_{i}&& \\
		&=1t_{1}+2t_{2}+3t_{3}+\cdots+nt_{n}&& \\
		&\text{buradan da,}&& \\
		&2a-2t_{2}-2t_{3}-\cdots=1t_{1}+1t_{3}+\cdots&&
	\end{flalign*}
\end{document}