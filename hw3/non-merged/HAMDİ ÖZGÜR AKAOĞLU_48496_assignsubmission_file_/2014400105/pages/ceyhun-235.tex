%\documentclass[fleqn]{book}
\documentclass[11pt]{amsbook}

\usepackage[turkish]{babel}

%\usepackage{../HBSuerDemir}	% ------------------------
\usepackage{../Ceyhun}	% ------------------------
\usepackage{../amsTurkish}


\begin{document}
% ++++++++++++++++++++++++++++++++++++++
\hPage{235}
% ++++++++++++++++++++++++++++++++++++++

	Eğer $d_1$ ve $d_3, Ç_2$'nin aynı parçası içindeyse, $Ç_1$'de yalnız {\large \textcircled{\small 1}} ve {\large \textcircled{\small 3}}  ile boyalı
	düğümlerden oluşan bir $Y_{13}$ yolu vardır. $d_{1}$, $ d_{0}$, $ d_{3}$ düğümlerinin oluşturduğu yol ile $Y_{13}$ yolu $Ç$ de, $d_2$ ya da $d_4$ ve
	$d_5$i çevreleyen bir çevre oluşturacaktır. $Ç_1$ de {\large \textcircled{\small 1}} ve {\large \textcircled{\small 4}} ile boyalı düğümlerin irgittiği
	altçizge $Ç_3$ olsun. Eğer $d_1$ ve $d_4$, $Ç_3$ ün ayrı parçalarında ise, $d_1$ in bulunduğu parçadaki düğümlerin renklerini değiştirirsek $d_0$a bitişik düğümlerin boyanmasında {\large \textcircled{\small 1}} 
	kullanılmamış olacaktır. Öyleyse, $d_0$'ı {\large \textcircled{\small 1}} e boyayabiliriz ve $Ç(d,a)$ 4-boyanırdır.
	
	Eğer $d_1$ ve $d_4$, $Ç_3$ ün aynı parçası içindeyse, $Ç_1$ de yalnız {\large \textcircled{\small 1}} ve {\large \textcircled{\small 4}} ile boyalı düğümlerden oluşan bir $ Y_{14}$
	yolu vardır. $d_1$ , $d_0$ ,$d_4$ düğümlerinin oluşturduğu yol ile $Y_{14}$ yolu $Ç$de $d_5$ ya da $d_2$ ve $d_3$ü çevreleyen bir çevre oluşturacaktır. Yukarda sözünü ettiğimiz her iki 		çevre de $Ç_1$ içinde yalnız {\large \textcircled{\small 2}}  ve {\large \textcircled{\small 4}}  ile boyanmış bir $Y_{24}$ ya da yalnız {\large \textcircled{\small 2}}  e {\large 					\textcircled{\small 3}}  ile boyanmış bir $Y_{35}$  yolu bulunmadığını önermez. $Ç_1$ de {\large \textcircled{\small 2}}  ve {\large \textcircled{\small 4}}  ile boyalı düğümlerin irgittiği 			altçizge $Ç_2$, {\large \textcircled{\small 2}}  ve {\large \textcircled{\small 3}} ile boyalı düğümlerin irgirriği altçizge ise $Ç_3$ olsun. Bu durumda $d_2$ ve $d_4$, $Ç_2$nin ayrı				parçalarında, $d_3$ ve $d_5$ ise $Ç_3$ ün ayrı parçalarında bulunacaktır. $Ç_2$ de, $d_2$nin bulunduğu parçadaki düğümlerin ve $Ç_3$ de de , $d_5$in bulunduğu parçadaki düğümlerin 		renklerini değiştirirsek, $Ç$ de $d_0$ a bitişik

\end{document}