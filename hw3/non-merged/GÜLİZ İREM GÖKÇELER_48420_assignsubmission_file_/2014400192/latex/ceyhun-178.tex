\documentclass[12 pt]{article}
\usepackage[utf8]{inputenc}
\usepackage[english]{babel}
\usepackage{enumerate}
\usepackage{amsthm}
\usepackage{amsmath}
\usepackage{amsfonts}

\begin{document}

\begin{proof}
We prove the lemma by the principle of mathematical induction. \\
We put\\
$ p_1 = q_1 $\\
$ p_k = q_1 $ and $ q_2 $ and $ \cdots $ and $ q_k $  (for all k $ \in \mathbb{N}, k \ge 2 )$.\\
Now induction.
\begin{enumerate}[I.]
  \item $ p_1 $ is true (by the hypothesis i.)
  \item Make the inductive hypothesis that $ p_k $ is true. Then $ q_1 $ and $ q_2 $ and $ \cdots $ and $ q_k $ is true. (definition of $ p_k $)\\ 
  $ q_1, q_2, \cdots , q_k $ are all true. (truth value of conjunction)\\
  $ q_{k+1} $ is true. (by the hypothesis ii.)\\
  $ q_1, q_2, \cdots , q_k, q_{k+1} $ are all true. \\
  $ q_1 $ and $ q_2 $ and $ \cdots $ and $ q_k $ and $ q_{k+1} $ is true.\\
  $ p_{k+1} $ is true.\\ 
  
\end{enumerate}
Hence, for all $ k \in \mathbb{N} $, if $ p_k $ is true, then $ p_{k+1} $ is true. By the principal of mathematical induction, $ p_n $ is true for all $ n \in \mathbb{N} $. So\\
$ q_1 $ and $ q_2 $ and $ \cdots $ and $ q_n $ is true for all $ n \in \mathbb{N} $.\\
In particular, $ q_n $ is true for all $ n \in \mathbb{N} $. This completes the proof. 
\end{proof} 

We can now formulate a new form of the principle of mathematical induction. This form will be used many times in the sequel. 

\subsection{Principle of mathematical induction:}
Let $ q_n $ be a statement-involving a natural number $ n $. We can prove the proposition for all $ n \in \mathbb{N}, q_n $\\
by establishing that

\begin{enumerate}[i.]
  \item $ q_1 $ is true
  \item for all $ k \in \mathbb{N}, $ if $ q_1, q_2, \cdots , q_k $ are true then, $ q_{k+1} $ is true. 
\end{enumerate}

The statement $ 2^n \ge n^2 $ is not true for all natural numbers $ n $, but true for all natural numbers $ n \ge 5 $. The principal of mathematical induction can be used to prove this and similar propositions. Let $ a $ be a fixed integer (positive, negative or zero) and let $ p_n $ be a statement involving an integer $ n \ge a $. We prove the truth of $ p_n $ for all $ n \ge a $ by showing that 

\end{document}