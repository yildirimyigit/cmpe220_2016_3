\documentclass{article}
\usepackage{amsmath, amsthm, amssymb}
\usepackage{../HBSuerDemir}
\begin{document}
	\hPage{feyzioğlu-075}
and these are verified easily. Hence
$ ({e,a,b }, \circ)$ is a group. There is a group of order 3. Any two groups of order 3 have essentialy the same Cayley table, namely the table in Figure 3. This statement will be made precise in §20.\\	

The Cayley tables of ( \hSoN ,+) and $({e,a,b}, \circ)$ are symmetric about the principal diagonal (that joins the upper-left and lower-right cells). What does this signify? The symmetry of the Cayley table of a group
$ (G, \circ)$means that the cell where the i-th and j-th row column meet, and this for all
$ i,j =1,2,...,|G|.$Assuming the i-th row is the row of $a \in G $and the j-th column is the column of $ b \in G$
(and assuming we index the rows and columns by the elements of G in the same order), this means:
$ a \circ b = b \circ a$for all  $a,b \in G. $
So the group is commutative in the following sense.
\begin{defn}
	A group 
	$(G, \circ)
	$ is called a \textit{commutative} group or an \textit{abelian} group, if, in addition to the group axioms 
	(i)-(iv) a fifth axiom \\
		(v)$ a \circ b = b \circ a
$
for all 
$a,b \in G.
$.
	holds.
\end{defn}
A binary operation on a set G is called \textit{commutative} when 
$ a \circ b = b \circ a
$
for all 
$a,b \in G.
$.
So a commutative group is one where the operation is commutative. The term "abelian" is used in honor of N. H. Abel, a Norwegian mathematician (1802-1829).
We close this paragraph with some comments on the group axioms. The reader might ask why we should study the structures 
$(G,\circ)$ where $\circ$ satisfies the axioms (i),(ii),(iii),(iv).Why do we not study structures $(G,\circ)$
where $\circ$ satisfies the axioms (i),(ii),(iii),(iv) to some other combination of (i),(ii),(iii),(iv)? There is of course no reason why other combinations ought to be excluded from study. As a matter of fact, all combinations have a proper name and there are theories about them. However, they are very far from having the same importance as the combination (i),(ii),(iii),(iv). 
\end{document}