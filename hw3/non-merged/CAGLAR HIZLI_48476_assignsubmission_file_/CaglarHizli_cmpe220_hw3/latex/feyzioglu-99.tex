\documentclass[11pt]{amsbook}

\usepackage[pdftex]{graphicx}
\usepackage{amsfonts}
\usepackage{enumerate}
\usepackage{../HBSuerDemir}	
\usepackage{epstopdf}
\epstopdfsetup{outdir=./images}
\epstopdfsetup{update} 
\setcounter{tocdepth}{3}

\usepackage{fancyhdr} 
\pagestyle{fancy}
\fancyfoot{}
\fancyfoot[L]{\footnotesize 
Freshman Calculus by Suer \& Demir  \textbf{DRAFT} \\
\LaTeX  by Haluk Bingol 
\href{http://www.cmpe.boun.edu.tr/~bingol}
{http://www.cmpe.boun.edu.tr/~bingol} 
\today}
\fancyfoot[R]{{\thepage} of \pageref{LastPage}}
	\usepackage[iso]{datetime}
	\newcommand{\hbTimeStamp}{{\color{red}--Draft-- v\today}} 
	\newcommand{\refcite}[1]{Ref~\cite{#1}}
	\newcommand{\hbQuote}[1]{{\small \textsf{``#1''}}}
	\newcommand{\reflst}[1]{List.~\ref{#1}}  
	\newcommand{\refthmA}[2]{\refthm{#1}(\ref{#2}}
	\newcommand{\refexmp}[1]{Example~\ref{#1}}
\usepackage{listings}             % Include the listings-package
\definecolor{hbGreen}{rgb}{0,0.6,0}
\definecolor{hbGray}{rgb}{0.5,0.5,0.5}
\definecolor{hbMauve}{rgb}{0.58,0,0.82}
\definecolor{hbbgColor}{rgb}{0.98,0.98,0.98}
%
\lstset{ %
	backgroundcolor=\color{hbbgColor},  
	basicstyle=\tiny\ttfamily, 
	breakatwhitespace=false,         % sets if automatic breaks should only happen at whitespace
	breaklines=true,                 % sets automatic line breaking
	captionpos=b,                    % sets the caption-position to bottom
	commentstyle=\color{hbGreen},    % comment style
	deletekeywords={...},            % if you want to delete keywords from the given language
	escapeinside={\%*}{*)},          % if you want to add LaTeX within your code
	extendedchars=true,              % lets you use non-ASCII characters; for 8-bits encodings only, does not work with UTF-8
	frame=single,                    % adds a frame around the code
	keepspaces=true,                 % keeps spaces in text, useful for keeping indentation of code (possibly needs columns=flexible)
	keywordstyle=\color{blue},       % keyword style
	language=Sh,                     % the language of the code
	morekeywords={*,...},            % if you want to add more keywords to the set
	numbers=left,                    % where to put the line-numbers; possible values are (none, left, right)
	numbersep=5pt,                   % how far the line-numbers are from the code
	numberstyle=\tiny\color{hbGray}, % the style that is used for the line-numbers
	rulecolor=\color{black},         
	showspaces=false,                % show spaces everywhere adding particular underscores; it overrides 'showstringspaces'
	showstringspaces=false,          % underline spaces within strings only
	showtabs=false,                  % show tabs within strings adding particular underscores
	stepnumber=2,                    % the step between two line-numbers. If it's 1, each line will be numbered
	stringstyle=\color{hbMauve},     % string literal style
	tabsize=1,                       % sets default tabsize to 2 spaces
	title=\lstname                   % show the filename of files included with \lstinputlisting; also try caption instead of title
	literate={├}{|}1 {─}{--}1 {└}{+}1
}
\lstset{basicstyle=\ttfamily}
	\newcommand{\hCode}[1]{{\small \texttt{#1}\,}}		
\title{
	Manual for\\
	\LaTeX\ Homeworks of cmpe220\\
	{\footnotesize \hbTimeStamp}
}
\author{Haluk O. Bingol}
\date{}                                           % Activate to display a given date or no date


\begin{document}
\hPage{feyzioglu-99}

(5) $Ha=Hb$ if and only if $a = hb$ for some $h \in H$, and there is a unique $h$ with $a=hb$, namely $h=ab^{-1}$ (Lemma 7.5(2)); thus $a=hb$ for some $h \in H$ if and only if $ab^{-1} \in H$. \par

(6)  $Ha=Hb$ if and only if $ab^{-1} \in H$ by (5), and $ab^{-1} \in H$ if and only if $Hab^{-1} = H$ by (2).\par

\textbf{10.3 Lemma}: \textit{Let} $H \leqslant G$. \textit{Then} $G$ \textit{is the union of the right cosets of} $H$. \textit{The right cosets of} $H$ \textit{are mutually disjoint. Analogous statements hold for left cosets.}\par
\textbf{Proof}: As $Ha\subseteq G$ for any $a \in G$, we get $\bigcup_{a \in G} Ha\subseteq G$. Also, for any $g \in G$, we have $g \in Hg$, so $g \in \bigcup_{a \in G} Ha$, thus $G \subseteq \bigcup_{a \in G} Ha$. This proves $G = \bigcup_{a \in G} Ha$. \par
Now we prove that the right cosets of $H$ are mutually disjoint. Assume $Ha \cap  Hb \neq \emptyset$. We are to show $Ha=Hb$. Well, we take $c \in Ha \cap  Hb $ if $Ha \cap  Hb \neq \emptyset$. Then $c \in Ha$ and $c \in Hb$. So $Ha=Hc$ and $Hc=Hb$ by Lemma 10.2(4). We obtain $Ha=Hb$. \par
The left cosets are treated similarly. \par
In the terminology of Theorem 2,5, right cosets of $H$ form a partition of $G$. Theorem 2.5 tells us that the right cosets are the equivalence classes of a certain equivalence relation on $G$. By the proof of Theorem 2.5, we see that this equivalence relation $\sim$ is given by 
\[ \text{for all } a,b \in G: a \sim b \text{ if and only if } Ha = Hb,\]
which we can read as
\[ \text{for all } a,b \in G: a \sim b \text{ if and only if } ab^{-1} \in Hb,\]
It may be worth while to obtain Lemma 10.3 from this relation $\sim$ instead of obtaining the relation $\sim$ from Lemma 10.3.\par
\textbf{10.4 Definition}: Let $H \leqslant G$ and $a,b \in G$. We write $a \equiv_r b \pmod{H}$ and say \textit{a is right congruent to b modulo H if} $ab^{-1} \in H$. Similarly, we write $a \equiv_l b \pmod{H}$ and say \textit{a is left congruent to b modulo H} if $a^{-1}b \in H$.

\end{document}  