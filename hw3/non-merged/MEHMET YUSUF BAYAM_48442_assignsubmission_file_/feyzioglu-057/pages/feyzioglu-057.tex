\documentclass[a4paper,11pt]{amsbook}
\usepackage{amsmath}
\usepackage{fullpage}
\usepackage{ tipa }
\pagestyle{headings}
\usepackage{../HBSuerDemir}

\begin{document}
\noindent
A rule of this type uses an auxiliary object $x$. The result then depends on
$a$ and $x$. At least,it seems so. This is due to the ambiguity in the second
step. This step states that we choose an $x$ with such and such property,
but there may be many objects $x,y,z,$... related to $a$ in the prescribed
manner. The auxiliary objects $x,y,z,$... will, in general, produce different
results. so we should perhaps that the result is $f(a,x)$ (or $f(a,y),f(a,z),$...).
In order the above rule to be a function, it must produce the same
result. Hence we must have $f(a,x)=f(a,y)=f(a,z)=\cdots$. The rule must be
so constructed that the same result will obtain even if we use different
auxiliary objects. If this be the case, the function is said to be \textit{well }
\textit{defined.}
\\
\\This terminology is somewhat unfortunate. It sounds as though there
are two types of functions, well defined functions and not well defined
functions (or badly defined functions). This is definitely not the case. A
well defined function is simply a function. Badly defined functions do
not exist. Being well defined is not a property, such as continuity,
boundedness, differentiability, integrability etc. that a function might or
might not possess. That a function $f:A\rightarrow B$ is well defined means: 1) the
rule of evaluating $f(a)$ for $a\in A$ makes use of auxiliary, foreign objects,
2) there are many choices of these foreign objects, hence 3) we have
reason to suspect that applying the rule with different choices may
produce different results, which would imply that our rule does not
determine $f(a)$ uniquely and $f$ is not a function in the sense of Definition
3.1, but 4)our suspicion is not justified, for there is a mechanism, hidden
under the rule, which ensures that same result will obtain even if we
apply the rule with different auxiliary objects. The question as to
whether a "function" is well defined arises only if that "function" uses
objects not uniquely determined by the element $a$ in its "domain" in
order to evaluate $f(a)$. We wrote "function" in quotation marks, for such
a thing may not be a function in the sense of Definition 3.1. Given such a
"function", which we want to be a function in the sense of Definition 3.1,
we check whether $f(a)$ is uniquely determined by $a$, that is, we check
whether $f(a)$ is independent of the auxiliary objects that we use for
evaluating $f(a)$. If this be the case, our supposed "function" $f$ is indeed a
function in the sense of Definition 3.1. We say then that $f$ is well defined,
of $f$ is a well defined function. This means $f$ is a function. In fact, it is
more accurate to say that a function is defined instead of saying that a
function is well defined.


\end{document}