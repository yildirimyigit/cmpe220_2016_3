%I used overleaf

\documentclass[11pt]{amsbook}

\usepackage{../HBSuerDemir}	

% ------------------------
\begin{document}
% ++++++++++++++++++++++++++++++++++++++
\hPage{feyzioglu-44}
% ++++++++++++++++++++++++++++++++++++++
	
will be called a \textit{composite} number. So $a \in \mathbb{Z}\setminus\{0\}$ is a composite number if and only if there is a $d \in Z$ with $1 < |d| < |a|$ and $d|a$.

Prime numbers are the building blocks of integers in the following sense.

	\begin{thm}
	Any nonzero integer, which is not a unit, is either a prime number or a product of prime numbers. 
	\end{thm}
	
    \begin{proof}
    Take an integer $n \neq 0$, and assume that $n$ is not a  unit. If $n$ is prime, there is nothing to prove. If $n$ is composite, then $n = n_1n_2$ for some $n_1,n_2 \in \mathbb{Z}, 1<|n_1|<|n|,1<|n_2|<|n|$. If $n_1$ and $n_2$ are prime, we are through. Otherwise, factor $n_1$ and $n_2$ into two numbers. Keep factoring until you get down to prime numbers. Since the factors get smaller and smaller in absolute value, we will teach prime numbers at the end. This is the basic idea and we make this reasoning into a rigorous proof by induction.

    We use Principle 4.5. Let $q_n$ be the statement that $n \in \mathbb{N}$ is a prime number or a product of prime numbers. We begin induction at $n = 2$. Since 2 is a  prime number, $q_2$ is true. $q_3$ is also true, for 3 is prime. $q_4$ is true, for 4 = 2.2 is a product of the prime numbers 2 and 2.

    Suppose now $q_2,q_3,q_4,...,q_{k-1}$ are true, so that 2,3,4,...,k-1 are either prime numbers or products of prime numbers. We want to prove that $k$ is a prime number or a product of prime numbers. If $k$ is prime, we are done. If $k$ is not prime, we have $k=k_1k_2,1<k_1<k,1<k_2<k$, for some integers $k_1,k_2$. Since $q_{k_1}$ and $q_{k_2}$ are true by the induction hypothesis, each of $k_l, k_2$ is either a prime number or a product of prime numbers:
    
		\begin{center}
		 $k_1'=p_1p_2...p_r \qquad k_2 =p_1'p_2'...p_s'$
		\end{center}

    where $p_1,p_2,...,p_r,p_1',p_2',...,p_s'$ are prime numbers ($r = 1$ or $s = 1$ is possible, in which case $k_1 = p_1$ or $k_2 = p_1'$ are prime numbers), and so
    
	    \begin{center}
		 $k = k_1k_2 = p_1p_2...p_rp_1'p_2'...p_s'$
		\end{center}
        
is a product of prime numbers. Hence $q_k$ is true.

This proves the theorem for positive integers. For a negative integer $-n$, where $-n$ is not a  unit, we have
    \end{proof}
% Proof continues in next page
\end{document}