%\documentclass[fleqn]{book}
\documentclass[11pt]{amsbook}

\usepackage[turkish]{babel}

%\usepackage{../HBSuerDemir}	% ------------------------
\usepackage{../Ceyhun}	
\usepackage{../amsTurkish}
\usepackage{tikz}

\newcommand*\circled[1]{\tikz[baseline=(char.base)]{
            \node[shape=circle,draw,inner sep=2pt] (char) {#1};}}

\begin{document}
% ++++++++++++++++++++++++++++++++++++++
\hPage{229}
% ++++++++++++++++++++++++++++++++++++++
yüzyılda 4 renk sanıtı tanıtlanmaya uğraşılırken, Kempe tanıtı yaptığını sandı ancak Heawood, Kempe'nin izlediği yolun yalnız 5-boyanırlığı 
tanıtlayabileceğini gösterdi. İlginçliklerinden dolayı Heawood'un 5-boyanırlık teoremini sonra da Kempe'nin yanlış teoremini vereceğiz.
\begin{theorem}
	Bütün düzlemsel çizgeler en çok 5-boyanırdır.
\end{theorem}
\begin{tanit}
	Düzlemsel bir $Ç(d, a)$ çizgesini düşünelim. d = 1, 2, 3, 4, ve 5 için Teoremin doğruluğu hemen görülebilir. Öyleyse Teoremin $d - 1$ 
	düğümlü bütün düzlemsel çizgeler için de doğru olduğunu varsayalım. \footnote{This theorem mentioned is on the page 178, That's why ref does not work properly} 
	Teorem 4.1.3\ref{thm:Teorem4.1.3} den, düzlemsel bir çizgede kertesi en çok 5 olan en az bir düğüm bulunduğunu biliyoruz.
	$Ç(d, a)$ da $d_0$, kertesi 5 olan bir  düğümü göstersin. 
	Yukarda yaptığımız varsayıma göre,
	\[ 
		Ç_1 = Ç(d,a) - (d_0)
	\]
	5-boyanır bir çizgedir. Kullanacağımız bu 5 ayrı rengi \circled{1}, \circled{2}, \circled{3}, \circled{4} ve \circled{5} diye simgeleyelim.
	Eğer $d_0$ a bitişik 5 düğümde, 5 ayrı renge boyanmamışsa, $d_0$ düğümü kullanılmayan renge boyayarak sorunu çözebiliriz. 
	Öyleyse, $d_0$ ın bitişik olduğu düğümler $d_0$, $d_1$, $d_2$, $d_3$, $d_4$, $d_5$, diye
	\footnote{This proof continues within the following page}
\end{tanit}
\end{document}