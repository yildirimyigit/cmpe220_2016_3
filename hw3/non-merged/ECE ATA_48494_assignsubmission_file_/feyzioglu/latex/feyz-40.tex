\documentclass[11pt]{amsbook}

\usepackage{../HBSuerDemir}
\begin{document}
    \hPage{Feyzioglu/40}
    
    Therefore, when we want to find $(a,b)$, we may assume $a \geqslant b$ without loss, of generality. (Instead of appealing to Theorem 5.6, we could use the definition to obtain $(a,b) = (-a,b) = (-a,-b) = (a,-b) = (b,a)$)
    
    The greatest common divisor of $a \in Z (a \ne 0)$ and $0$ is easily found. We have $(a,O) = \left | a \right \vert$, as follows from Theorem 5.6 or immediately from Theorem 5.4.
    
    Suppose now $a \geqslant b > 0$ and we want to find $(a,b)$. We divide $a$ by $b$ and get
    \[
        a = q_1 b + r_1 , 0 < r_1 < b. 
    \]
    
    Here $r_1$ may be zero. If $r_1 \ne 0$, we divide $b$ by $r_1$ and get 
    \[
        b = q_2 r_1 + r_2 , 0 \le r_2 < r_1. 
    \]
        
    Here $r_2$ may be zero. If $r_2 \ne 0$, we divide $r_1$ by $r_2$ and get 
    \[
        r_1 = q_3 r_2 + r_3 , 0 \le r_3 < r_2. 
    \]
    
    We proceed in this way. We have $b > r_1 > r_2 > r_3 > \dots$. Since the $r_y 's$ are nonnegative integers and $b$ is a finite positive integer, this process cannot go on indefinitely. Sooner or later, we will meet a division in which the remainder is zero, say at the $(k+l)$-st step $(k \ge 0)$:
    \begin{align*}
        &r_{k-2} = q_k r_{k-1} + r_k , 0 \le r_k < r_{k-1}  \\
        &r_{k-1} = q_{k+1} r_{k} + r_{k+1} , 0 = r_{k+1}
    \end{align*}
    
    We claim that $r_k$, the last nonzero remainder, is the greatest common divisor of $a$ and $b$, and that it can be written in the form $ax - by$, where $x$, $y$ are integers.
    
    \begin{thm}
        Let $a \ge b > 0$ be integers and let
        
        \begin{align*}
        &a = q_1 b + r_l,       &   &0 \le r_1 < b,      \\
        &b = q_2 r_1 + r_2,     &   &0 \le r_2 < r_1,    \\
        &r_1 = q_3 r_2 + r_3,   &   &0 \le r_3 < r_2,    \\
        &\cdots\cdots\cdots\cdots\cdots\cdots\cdots\cdots   \\
        &r_{i-1} = q_{i+1} r_i + r_{i+1},   &   &0 \le r_{i+1} < r_i,    \\
        &\cdots\cdots\cdots\cdots\cdots\cdots\cdots\cdots   \\
        &r_{k-2} = q_k r_{k-1} + r_k,   &   &0 \le r_k < r_{k-1},      \\
        &r_{k-1} = q_{k+1} r_{k}
        \end{align*}
    \end{thm}
\end{document}