\documentclass{article}
\usepackage{../HBSuerDemir}
\begin{document}


\hPage{Feyzioglu-10}
\begin{itemize}
    \item[(ii)]  $(a,b) \approx (c,d)$ then $ad=bc$, then $da=bc$, then $cb=da$, so $(c,d) \approx (a,b)$ 
    \item[(iii)] (a,b) $\approx (c,d)$ and (c,d) = (e,f), then 
  \end{itemize}
   \begin {equation}
 $$ad=bc   and    cf = de$$
 $$adf = cbf   and    bcf = bde$$
 $$adf = bde$$
 $$d(af-be) =0$$
 $$af-be = 0   (since $ d \neq 0) $$$
 $$af = be$$
 $$$(a,b) \approx (e,f)$$$ 
  \end {equation}
  Thus, $\approx$ is an equivalence relation on S.
  \begin{itemize}
  \item[(g)] Let T be the set of all triangles in the Euclidean plane. Congruence of triangles is 
  an equivalence relation on T.
  \item[(h)] Let S be the set of all continuous functions defined on the closed interval [0,1].
  For any two functions f,g in S, let us write $f\overset{\underset{\mathrm{A}}{}}{=}g$ if % buraya bak , a?a??da da var.
  \end{itemize}
  \begin {equation}
  $$ \[ \int_{0}^{1} f(x)dx = \int_{0}^{1}g(x)dx\] $$ 
  \end{equation}
  Then, $\overset{\underset{\mathrm{A}}{}}{=}$ is an equivalence relation on S. \\
  An equivalence relation is a weak form of equality. Suppose we have various objects, 
  which are similar in one respect and dissimilar in certain other respects. We may wish to 
  ignore their dissimilarity and focus our attention on their similar behaviour. Then there is 
  no need to distinguish between our various objects that behave in the same way. We may
  regard them as equal or identical. Of course, "equal" or "identical" are poor words to employ
  here, for the objects are not absolutely identical, they are equal only in one respect that we 
  wish to investigate more closely. So we employ the word "equivalent". That a and b are 
  equivalent means, then, a and b are equal, not in every respect, but rather as far as a 
  particular property is concerned. An equivalence relation is a formal tool for disregarding 
  differences between various objects and treating them as equals.
  Let us examine our examples under this light. In example 2.3(b) the points P are R may be
  different, but the lines they determine with the origin are equal. In Example 2.3(c), the lines
  may be different, but their directions are equal. In Example 2.3(d), the integers may be 
  different, but their parities are equal. In Example 2.3(e), the integers
\end{document}
