\documentclass[fleqn]{book}
\usepackage[utf8]{inputenc}
\usepackage{../HBMath}
\usepackage{../Ceyhun}

\begin{document}

çizgesi 4-boyanırdır. $Ç_1$ çizgesinde $d_1$ ve $d_3$
düğümlerinin irgittiği altçizgeyi $Ç_2$ ile
gösterelim. Eğer $d_1$ ve $d_3$, $Ç_2$ nin iki ayrı
parçasında ise, $d_1$ in bulunduğu parçadaki
düğümlerin rengini değiştirerek $d_0$ a bitişik
düğümlerin boyanmasında \textcircled{1} in kullanılmadığı
yeni bir boyama düzeni elde ederiz. öyleyse,
$d_0$ ı \textcircled{1} e boyayabiliriz ve $Ç(d,a)$ 4-boyanırdır.
\\
Eğer $d_1$ ve $d_3$ düğümleri, $Ç_2$ çizgesinin aynı
parçası içindeyse, yalnız \textcircled{1} ve \textcircled{3} ile boyanmış
düğümleri içeren bir $Y_{13}$ yolu vardır. $d_3, d_0, d_1$
düğümlerinin oluşturduğu yöl ile $Y_{13}$ yolu, $Ç(d,a)$
çizgesinde $d_2$ ya da $d_4$ düğümünü çevreleyen bir
çevre oluşturacaktır. Öyleyse, $Ç_1$ de yalnız \textcircled{2}
ve \textcircled{4} ile boyalı düğümleri içeren bir $Y_{24}$ yolu
yoktur. $Ç_1$ çizgesinde, \textcircled{2} ve \textcircled{4} boyanmış
düğümlerin irgittiği altçizgeye $Ç_3$ ile gösterelim.
$Ç_3$ parçalı bir çizgedir ve $d_2$ ile $d_4$ bu çizgenin
iki ayrı parçasındadır. $d_2$ nin bulunduğu parçadaki
düğümlerin rengini değiştirirsek, $d_0$ a bitişik
düğümlerin boyanmasında \textcircled{2} kullanılmamış olacaktır.
Öyleyse, $d_0$ ı \textcircled{2} ye boyayabiliriz ve $Ç(d,a)$
4-boyanırdır.
\\
\textit{Durum 2:} $d_0$ ın kertesi 5 olsun.
\\
$d_0$ a bitişik düğümleri $d_1, d_2, d_3, d_4$ ve $d_5$ diye
simgeleyelim. $Ç$ dönüşül düzlemsel olduğu için, bu

\end{document}