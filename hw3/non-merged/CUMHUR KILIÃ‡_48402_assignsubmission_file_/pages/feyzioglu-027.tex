\documentclass[11pt]{amsbook}

\usepackage{../HBSuerDemir}	% ------------------------


\begin{document}

% ++++++++++++++++++++++++++++++++++++++
\hPage{feyzioglu/027}
We know introduce binary operations.They constitute a generalization of the four elementary operations addition, subtraction, multiplication and
division that everybody learns in the primary school. Consider addition, for example. Given any two numbers a and b, their sum is a uniquely determined number. This is the core of the operation concept: given two
objects a and b, associate with them a unique object of the same kind. More precisely, we have the 
\begin{defn}
    Let S  be a non empty set. A binary operation on S is a mapping from $ S \times S $ into S. \\
    The important things about a binary operation $ \lambda$
    is that it is defined for all ordered pairs $ \left( a,b \right ) \in  S \times S $ and that the result of operation, $ (a,b)\lambda$, is an element of S . \\
    Although a binary operation $ \lambda$ is a mapping, we will not employ the functional notation $(a,b)\lambda $.
    As in the case of the elementary operations, we write a sign like $ "+","-","\circ","\oplus","\otimes"$ between the elements a and b to denote the image of $(a,b)$ under $\lambda$. So the image of $(a,b)$ will be denoted
    by $a+b,a-b,a\circ b, a \oplus b, a \otimes b$ or by a similar symbol.
\end{defn}
\begin{exmp}
(a)The elementary operations addition, subtraction, multiplication are binary operations on $\hSoR$. Subtraction is not a binary operation on  $\hSoN$, since 1 -2, for instance, is not an element of  $\hSoN$ (although 1 and 2 are).\\
(b) Let M be a set and let S be the set of all subsets of M. Taking union and taking intersection are binary operations on S. The usual notation $" A \hUnion B ", " A \hIntersection B" $ conforms to the remarks above. \\
(c)Let F be the set of all functions from a set A into A. The usual composition of functions is a binary operation on F. \\
(d)Let us write $x \circ y = x + y^2 $ and $ x \bigtriangleup y = x^2+x+1$ for real numbers x,y. Then
$\circ$ and $\bigtriangleup$ are binary operations on $\hSoR$. Here y does not enter into $x \circ y $ in any way, but this does not preclude $\bigtriangleup$ from being a binary operation. \\
(e)Let V be the set of all vectors in the tree space 
$\hSoR^3$. Taking dot product of two vectors is not a binary operation on V, since the result is 
\end{exmp}


\end{document}