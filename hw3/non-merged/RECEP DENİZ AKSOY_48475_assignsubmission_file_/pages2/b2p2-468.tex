\documentclass[11pt]{amsbook}

\usepackage{../HBSuerDemir}	% ------------------------


\begin{document}
\hPage{b2p2/468}
The first of which gives a parametric solution
\begin{center}
x = -\operatorname*{\psi\prime}(p)\\
y = xp + \operatorname*{\psi}(p)\\
\end{center}
This solution not involving an arbitrary constant cannot the $GS$. The second gives $p = c$ or the solution
\begin{center}
y = cx + \operatorname*{\psi}(c)\\
\end{center}
which the $GS$.\\
The solution $(2)$ cannot be obtainable from the $GS$ $(3)$ by a choice of the arbitrary constant $c$. Hence $(2)$ is not a solution. It is called a $singular$ $solution$ of $(1)$. The graph of this singular solution can be seen to be envelope of the family $(3)$.\\
Summarizing, the $GS$ of $(1)$ is obtained by replacing $y\prime$ by an arbitrary constant in the DE and the envelope of $(3)$ is the singular solution.\\
\begin{exmp}
Find the singular and general solution of the $CDE$:
\end{exmp}
\begin{center}
y = xy\prime + \frac{1 + y\prime}{1 - y\prime}\\
\end{center}
\begin{hSolution}
The $GS$ is obtained by replacing $y\prime$ by $c$:
\begin{center}GS:\quad y = cx + $\frac{1 + c}{1 - c}$\\SS:\quad x = $\frac{-2}{(1 - p)^2}$, y = xp + $\frac{1+p}{1-p}$\\
\end{center}
2. LAGRANGE Differential Equation by replacing $y\prime$ by $c$:
\begin{center}
y = x \phi(y\prime) + \psi(y\prime), \quad (\phi(y\prime)\not\equiv y\prime)\\
\end{center}
Setting $y\prime = p$ in $(1)$ and considering $p$ as the new variable, differentiation of $(1)$ gives
\begin{center}
p = \phi(p) + x\phi\prime(p)\frac{\hDif p}{\hDif x} + \psi\prime(p)\frac{\hDif p}{\hDif x}\\
\end{center}
\end{hSolution}

\end{document}  

