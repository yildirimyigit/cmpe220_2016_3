\documentclass[11pt]{amsbook}

\usepackage{../HBSuerDemir}
	\begin{document}
		\hPage{feyzioğlu-059}
	and \hDefined{b)}. 2) Evaluate $F(a)$ and $F(b)$. 3) Put $I(f) = F(b) - F(a)$. There are many functions $F$ with $F'(x) = f(x)$ for all $x$ $\in$ $\lbrack a,b\rbrack $. For two different choices $F_1$ and $F_2$, we have $F_1(b) \neq F_2(b)$ and $F_1(a) \neq F_2(a)$ in general. So we may suspect that $F_1(b) - F_1(a) \neq F_2(b) - F_2(a)$. In order to show that $I$ is a well defined function, we must prove $F_1(b) - F_1(a) = F_2(b) - F_2(a)$ whenever $F_1$ and $F_2$ are functions on $\lbrack a,b \rbrack$ such that $F_1'(x) = f(x) = F_2'(x) $ for all $x \in \lbrack a,b \rbrack$. We know from the calculus that, when $F_1$ and $F_2$ have this property, there is a constant $c$ such that $F_1(x) = F_2(x) + c$ for all $x \in \lbrack a,b \rbrack$. So $F_1(b) - F_1(a) = (F_2(b) + c) - (F_2(a) + c) = F_2(b) - F_2(a)$. Therefore, $I$ is well defined. \\
	
	After this lengthy digression, we return to the integers mod $n$ and to the "operations" $\bigoplus$ and $\bigotimes$ .
	\begin{lem} 
		$\bigoplus$ and $\bigotimes$ are well defined operations on $\mathbb{Z}_n$ \\
	\end{lem}  
		\begin{proof}
			We are to prove $\bar{a}$ $\bigoplus$ $\bar{b}$ $=$ $\bar{a}'$ $\bigoplus$ $\bar{b}'$ whenever $\bar{a}$ $=$ $\bar{a}'$ and $\bar{b}$ $=$ $\bar{b}'$ in $\mathbb{Z}_n$ (different names for identical residue classes should not yield different results.) This follows from \hDefined{Lemma 6.1}. Indeed, if $\bar{a}$ $=$ $\bar{a}'$ and $\bar{b}$ $=$ $\bar{b}'$, then $a \equiv a'(mod n)$   and $b \equiv b'$(mod $n$) by definition, so we obtain $a +b = a' + b'$(mod $n$) and $ab = a'b'$(mod $n$) by \hDefined{Lemma 6.1}, hence  $\overline{a+b}$ $=$ $\overline{a'+b'}$  and  $\overline{ab}$ $=$ $\overline{a'b'}$  , which gives $\bar{a}$ $\bigoplus$ $\bar{b}$ $=$ $\overline{a+b}$ $=$ $\overline{a'+b'}$ $=$ $\bar{a}'$ $\bigoplus$ $\bar{b}'$ and $\bar{a}$ $\bigotimes$ $\bar{b}$ $=$ $\overline{ab}$ $=$ $\overline{a'b'}$ $=$ $\bar{a}'$ $\bigotimes$ $\bar{b}'$.
		\end{proof}
		Having proved that $\bigoplus$ and $\bigotimes$ are well defined operations on $\mathbb{Z}_n$ , we proceed to show that $\bigoplus$ and $\bigotimes$ possess many (but not all) properties of the usual addition ad multiplication of integers. First we simplify our notation. From now on, we write $+$ and $.$ instead of $\bigoplus$ and $\bigotimes$ . In fact, we shall even drop and use simply juxtaposition to denote a product of two integers mod $n$. Thus we will have $\bar{a}$ $+$ $\bar{b}$ $=$ $\overline{a+b}$ and $\bar{a}$ $.$ $\bar{b}$ $=$ $\overline{ab}$ or simply $\bar{a}$  $\bar{b}$ $=$ $\overline{ab}$. The reader should note that the same sign "+" is used to denote two very distinct operations: $\bigoplus$ in the old notation and the usual addition of integers. If anything, they are defined on distinct sets $\mathbb{Z}_n$ and $\mathbb{Z}$. The same remarks apply to multiplication.
	
	\end{document}