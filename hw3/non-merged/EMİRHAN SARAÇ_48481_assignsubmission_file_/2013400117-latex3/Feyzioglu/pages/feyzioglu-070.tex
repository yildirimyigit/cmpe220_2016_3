\documentclass[11pt]{amsbook}

\usepackage{../HBSuerDemir}

\begin{document}

    \hPage{70}

    \begin{enumerate}
        \item[3)] that $e$ is the unique element of $G$ with these two properties,
        \item[4)] that for each $a\in G$, there is an $a^{-1}\in G$ such that $a\circ a^{-1}=e$,
        \item[5)] that $a^{-1}\circ a=e$ as well, 
        \item[6)] that this $a^{-1}$ is the unique element of $G$ with $a\circ a^{-1}=e =a^{-1}\circ a$,
    \end{enumerate}
    
    \begin{flushleft}
        which more than doubles our work. With our Definition 7.2, we need check only 1) and 4). The other items 2), 3), 5), 6) follow from 1) and 4) automatically. We pay for our comfort by having to prove Lemma 7.3, but, once this is over, we have less work to do in order to see whether a given set $G$ forms a group under a given operation $\circ$ on it, as in the following examples.
    \end{flushleft}
    
    \begin{flushleft}
        7.4 \textbf{Examples: (a)} For any two elements $a, b$ of $\hSoQ\setminus \{1\}$, we put $a\circ b=ab -a -b +2$. We ask if $\hSoQ\setminus\{1\}$ is a group under $\circ$. Let us check the group axioms.
    \end{flushleft}
    
    \hspace{15mm}(i) For all $a,b\in\hSoQ\setminus\{1\}$, we observe $a\circ b=ab -a -b +2\in\hSoQ$, but this is not enough. We must prove $a\circ b\neq 1$ also. Let $a,b\in\hSoQ, a\neq 1\neq b$. We suppose $a\circ b= 1$ and try to reach a contradiction. If $a\circ b= 1$, then
    
    \vspace{-5mm}
    
    \begin{align*}
        ab-a&-b+2=1 \\
        ab-a&-b+1=0 \\
        (a-1)&(b-1)=0 \\
        a-1=0 &\text{ or } b-1=0 \\
        a=1 &\text{ or } b=1,
    \end{align*}
    
    \begin{flushleft}
        a contradiction. So $a\circ b\in\hSoQ\setminus\{1\}$ and $\circ$ is a binary operation on $\hSoQ\setminus\{1\}$.
    \end{flushleft}
    
    \hspace{15mm}(ii) For all $a,b,c\in \hSoQ\setminus\{1\}$, we ask if $(a\circ b)\circ c=a\circ (b\circ c)$. 
    
    \vspace{-5mm}
    
    \begin{align*}
        (ab-a-b+2)\circ c &\stackrel{?}{=} a\circ (bc-b-c+2) \\
        (ab-a-b+2)c-(ab-a-b+2)-c+2 &\stackrel{?}{=} a(bc-b-c+2)-a-(bc-b-c+2)+2 \\
        abc-ac-bc+2c-ab+a+b-2-c+2 &\stackrel{?}{=} abc-ab-ac+2a-a-bc+b+c-2+2
    \end{align*}
    
    \begin{flushleft}
        The answer is "yes." So $\circ$ is associative.
    \end{flushleft}
    
    \hspace{15mm}(iii) We are looking for an $e\in\hSoQ\setminus\{1\}$ such that $a\circ e=a$ for all $a\in\hSoQ\setminus\{1\}$. Assuming such an $e$ exists, we get
    
    \vspace{-5mm}
    
    \begin{align*}
        ae-a-e+2&=a \\
        ae-a&=2a-2 \\
        (a-1)e&=2(a-1) \\
        e&=2 \text{\qquad (since $a-1\neq0$).}
    \end{align*}
    
    \begin{flushleft}
        We have not proved that $2\in\hSoQ\setminus\{1\}$ is a right identity element. We showed only that a right identity element, if it exists at all, has to be 2. Let us see if 2 is really a right identity. We observe
    \end{flushleft}
    
\end{document}  
