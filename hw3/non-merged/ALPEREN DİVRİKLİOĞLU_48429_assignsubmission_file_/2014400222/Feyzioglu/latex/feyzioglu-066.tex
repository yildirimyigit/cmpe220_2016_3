\documentclass[11pt]{amsbook}

\usepackage{../HBSuerDemir}

\begin{document}

\hPage{feyzioglu-066}

\begin{lem}
Let $(G, \circ)$ be a group and let $e$ be a right identity element of $G$ such that, for all $a$ $\epsilon$ $G$, there exists a suitable $x$ in $G$ with $a \circ x = e$. The existence of $e$ is assured by the group axioms \ref{III} and \ref{IV}.
\end{lem}

\begin{ax}
If $g$ $\epsilon$ $G$ is such that $g \circ g = g$, then $g = e$.
\end{ax}

\begin{ax}
$e$ is the unique right identity in $G$.
\end{ax}

\begin{ax}
\label{III}
A right inverse of an element in $G$ is also a left inverse of the same element. In other words, if $a \circ x = e$, then $x \circ a = e$.
\end{ax}

\begin{ax}
\label{IV}
$e$ is a left identity in $G$. That is, $e \circ a = a$ for all $a$ $\epsilon$ $G$.
\end{ax}

\begin{ax}
$e$ is the unique left identity in $G$.
\end{ax}

\begin{ax}
Each element has a unique right inverse in G.
\end{ax}

\begin{ax}
Each element has a unique left inverse in G.
\end{ax}

\begin{ax}
The unique right inverse of any $a$ $\epsilon$ $G$ is equal to the unique left inverse of $a$.
\end{ax}

\begin{proof}
(1) Let $g$ $\epsilon$ $G$ be such that $g \circ g = g$. We choose a right inverse of g with respect to $e$. This is possible by the axiom \ref{IV}. Let us call it $h$. Thus $g \circ h = e$. Then
\[
(g \circ g) \circ h = g \circ h
\]
\[
g \circ (g \circ h) = g \circ h \text{ (by associativity),}
\]
\[
g \circ e = e \text{ (since $g \circ h = e$),}
\]
\[
g = e \text{ (since e is a right identity).}
\]

This proves part(1).

(2) The claim is that $e$ is the unique right identity in $G$. This means: if $f$ $\epsilon$ $G$ is a right identity, that is, if $a \circ f =  a$ for all $a$ $\epsilon$ $G$, then $f = e$. Suppose $f$ is a right identity. Then $a \circ f = a$ for all $a$ $\epsilon$ $G$. Writing $f$ for $a$ in particular, we see $f \circ f = f$. Hence $f = e$ by part (1).

(3) A right inverse $x$ of an arbitrary element $a$ $\epsilon$ $G$ is also a left inverse of $a$. This is what we are to prove. So we assume $a \circ x = e$ and try to derive $x \circ a = e$. We use part(1). If $a \circ x = e$, then
\[
(x \circ a)\circ(x \circ a) = [(x \circ a ) \circ x] \circ a \text{ (by associativity)}
\]
\[
= [x \circ (a \circ x)] \circ a \text{ (by associativity)}
\]
\[
= [x \circ e] \circ a
\]
\[
= x \circ a. 
\]
So $g := (x \circ a)$ is such that $g \circ g = g$. By part (1), $g = e$. So $x \circ a = e$.

(4) We are to prove that $e$ is a left identity. So we must show $e \circ a = a$ for all $a$ $\epsilon$ $G$. Let $a$ $\epsilon$ $G$ and let $x$ be a right inverse of $a$. Then
\[
a \circ x = e.
\]
\end{proof}

\end{document}