\documentclass[11pt]{amsbook}

\usepackage{../HBSuerDemir}	% ------------------------

\begin{document}
% ++++++++++++++++++++++++++++++++++++++
\hPage{feyzioglu-039}.
% ++++++++++++++++++++++++++++++++++++++

$ d | d' $ by (ii). By Lemma 5.2(12), we obtain $|d| = |d'|$. From (iii), we get $ d > 0 $, $d' > 0$, which yields $d = d'$. Thus $d$ is unique.

\begin{defn}
Let $ a,b \ \epsilon \ \mathbb{Z}$, not both zero: The unique integer $d$ in Theorem 5.4 is called the \emph{greatest common divisor of $a$ and $b$}.


The greatest common divisor of $a$ and $b$ will be denoted by $(a ,b)$. This notation is standard. The reader should not confuse it with an ordered pair. The greatest common divisor of $a$ and $b$ is a natural number, not an ordered pair.


Definition 5.5 and the proof of Theorem 5.4 enables us to write the
\end{defn}

\begin{thm} Let $ a,b \ \epsilon \ \mathbb{Z}$, not both zero. Then $(a,b)$ is the smallest positive integer in the set $\{ ax - by \ \epsilon \ \mathbb{Z}: x,y \ \epsilon \ \mathbb{Z}\}$.
\end{thm}


Theorem 5.4 is a typical existence theorem. It tells us that the greatest common divisor $(a,b)$ of any pair of integers $a$, $b$ exists (provided a and b are not both zero), but gives no method for finding it. If $a$ and $b$ are small in absolute value, we might try to find the smallest positive integer in the set $\{ ax - by \ \epsilon \ \mathbb{Z}: x,y \ \epsilon \ \mathbb{Z} \}$. This is not very satisfactory, of course. Also, it is almost impossible if $a$ and $b$ are rather large. We propose to give a systematic method for finding $(a, b)$ for any pair of integers $a$, $b$, not both zero. This method will prove anew the existence of $(a,b)$ and in addition will give us a systematic method of finding integers $x$, $y$ such that $(a,b) = ax - by$. It is Proposition 2 in Euclid's \emph{Elements}, Book
VII. (in algebraic notation) and is known as the Euclidean algorithm.


We first observe that the set $U$ in Theorem 5.6 does not change if we write $-a$ in place of $a$ or $-b$ in place of $b$. This yields
\[
    (a,b) = (-a,b) = (-a,-b) = (a,-b)
\]
for all $a$, $b$, not both zero. Hence $(a,b) = (|a|,|b|)$ and, when we want to find $(a, b)$, we may assume $a \geq 0$, $b \geq 0$ (the case $a = 0$, $b = 0$ is excluded) without loss of generality. Moreover, the set $U$ in Theorem 5.6 remains unaltered if we interchange $a$ and $b$. Thus
\[
(a,b) = (b,a)
\]    
% =======================================================
\end{document}