\documentclass{amsbook}
\usepackage{../HBSuerDemir}
\usepackage{tikz}
\usepackage{dcolumn}
\usepackage{fixltx2e}
\newcolumntype{2}{D{.}{}{2.0}}

\begin{document}
\hPage{feyzioglu/73}
As an illustration, we give the addition table of $(\mathbb{Z}_{4}, +)$ below. $(\mathbb{Z}_{4}, +)$ is a group by Example 7.1(c). We drop the bars for convenience. 
\begin{center}\footnote{used tikz, dcolumn and fixltx2e packages}
\renewcommand\arraystretch{1.3}
\setlength\doublerulesep{0pt}
\begin{tabular}{r||*{4}{2|}}
+ & 0 & 1 & 2 & 3 \\
\hline\hline
0 & 0 & 1 & 2 & 3 \\ 
\hline
1 & 1 & 2 & 3 & 0 \\ 
\hline
2 & 2 & 3 & 0 & 1 \\ 
\hline
3 & 3 & 0 & 1 & 2 \\ 
\hline
\end{tabular}
\end{center}

We observe in this table that every element of $\mathbb{Z}_{4}$ appears once in each row and also in each column. This is a general property of groups: if $(G, \circ)$ is a group, then every element $b$ of $G$ appears once and only once in the row of any $a \in G$, say in the cell where the row of a and the column of $x \in G$ meet. A similar assertion holds for columns. This is the content of the next lemma.

\begin{lem}
Let $(G, \circ)$ be a group and $a, b \in G$.\\
(1) There is one and only one $x \in G$ such that $a \circ x = b$.\\
(2) There is one and only one $y \in G$ such that $y \circ a = b$.
\end{lem}

\begin{proof}
(1) We prove first that there can be at most one $x \in G$ such that $a \circ x = b$. Let $a \circ x = b = a \circ x_{1}$. We prove $x = x_{1}$. We have
\begin{align*}
a \circ x &= a \circ x_{1}\\
a^{-1} \circ (a \circ x) &= a^{-1} \circ (a \circ x_{1})\\
(a^{-1} \circ a) \circ x &= (a^{-1} \circ a) \circ x_{1}\\
e \circ x &= e \circ x_{1}\\
x &= x_{1}
\end{align*}
by Lemma 7.3. So there can be at most one x with $a \circ x = b$.\\
The. existence of at least one such $x$ is easily seen when we put $x = a^{-1} \circ b$. Indeed, $a \circ (a^{-1} \circ b) = (a \circ a^{-1}) \circ b = e \circ b = b$.\\
So there is one and only one element $x$ of $G$, namely $x = a^{-1} \circ b$, such that $a \circ x = b$. This proves (1).\\
The proof of (2) is similar and is left to the reader.
\end{proof}
\end{document}
