\documentclass[11pt]{amsbook}

\usepackage{../HBSuerDemir}	% ------------------------


\begin{document}

% ++++++++++++++++++++++++++++++++++++++
\hPage{p1/11}
% ++++++++++++++++++++++++++++++++++++++

\noindent different, but their remainders, when they are divided by $n$, are equal.
In Example 2.3(f), the pairs may be different, but the ratio of their
components are equal. In Example 2.3(g), the triangles may have
different locations in the plane, but their geometrical properties are the same. In Example 2.3(h), the functions may be different, but the "areas under their curves" are equal. \par
An equivalence relation $\sim$ on a set $A$ gives rise to a partition of $A$ into disjoint subsets. This means that $A$ is a union of certain subsets of $A$ and that the distinct subsets here are mutually disjoint. The converse is also true: whenever we have a partition of a nonempty set $A$ into pairwise disjoint subsets, there is an equivalence relation on $A$. Before proving this important result, we introduce a definition. \par
\begin{defn} Let $\sim$ be an equivalence relation on a nonempty set $A$,
and let $a$ be an element of $A$. The equivalence class of a is defined to be the set of all elements of $A$ that are equivalent to $a$.  \par
The equivalence class of $a$ will be denoted by $(a]$ (or by class$(a)$,$cl(a)$,$\bar{a}$ or by a similar symbol): $[a] = \{x \in A : x \sim a \}$.
\end{defn}
An element of an equivalence class $X \subseteq A$ is called a $representative of X$. Notice that $x \in [a]$ and $x \sim a$ have exactly the same meaning. In particular, we have $a \in [a]$ by reflexivity. So any $a \in A$ is a representative of its own equivalence class. \par
The equivalence classes $[a]$ are subsets of $A$. The set of all equivalence classes is sometimes denoted by $A/{\sim}$. It will be a good exercise for the reader to find the equivalence classes in Example 2.3. \par
We now state and prove the result we promised.\par
\begin{thm}
Let $A$ be a nonempty set and let $\sim$ be an equivalence 
relation on $A$. Then the equivalence classes form a partition of $A$. In
other words, $A$ is the union of the equivalence classes and the distinct equivalence classes are disjoint:
\end{thm} 


\end{document}
