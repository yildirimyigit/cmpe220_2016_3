\documentclass[10pt, twoside, a4paper]{article}

\usepackage{amssymb}
\usepackage{amsmath}
\usepackage{amsfonts}
\usepackage{amsthm}
\usepackage{multicol}
\usepackage{amsmath}
\usepackage{amssymb}
\newcommand*{\field}[1]{\mathbb{#1}}
\usepackage{natbib}

\usepackage{geometry}
\usepackage{fancyhdr}
\usepackage{amssymb}
\usepackage{amsmath}
\numberwithin{equation}{section}
\numberwithin{figure}{section}
\usepackage{amsfonts}
\usepackage{amsthm}

\setcounter{page}{32}

\begin{document}

(b) Prove that  $2 + 2^2 + 2^3 +$ \dots $+ 2^n = 2^{n+1} - 2 $ for all $n \in \field{N}$. \vspace{0.2cm}

\hspace{2cm} \textbf{I.} We have $2 = 2^{1+1} -2$, which proves the assertion for $n=1$. \vspace{0.2cm}

\hspace{2cm} \textbf{II.}Assume  $2 + 2^2 + 2^3 +$ \dots $+ 2^k = 2^{k+1} - 2 $. Now we must prove \vspace{0.2cm}

$2 + 2^2 + 2^3 +$ \dots $+ 2^k + 2^{k+1} = 2^{(k+1)+1} -2$. We have \vspace{0.2cm}

\hspace{2cm} $2 + 2^2 + 2^3 +$ \dots $+ 2^k + 2^{k+1} = (2^{k+1} -2) + 2^{k+1} $     (by inductive hyp.)  \vspace{0.2cm}

\hspace{2cm} $= 2(2^{k+1}) - 2$ \vspace{0.2cm}

\hspace{2cm} $= 2^{k+2} - 2$, \vspace{0.2cm} 

so the assertion is true for $n = k+1$ if it is true for $n = k$. Thus \vspace{0.2cm}

\hspace{2cm} $2 + 2^2 + 2^3 +$ \dots $+ 2^n = 2^{n+1} - 2 $ for all $n \in \field{N}$.  \vspace{0.2cm}

(c) Let $h > -1$ be a fixed real number. Prove that ${1+h}^n \ge 1+nh$ for all $n \in \field{N}$.  \vspace{0.2cm}

\hspace{2cm} \textbf{I.}  We have ${(1+h)}^1 \ge 1+1h$, so the inequality is true for $n=1$. \vspace{0.2cm}

\hspace{2cm} \textbf{II.} Let us assume ${(1+h)}^k \ge 1+kh$. We want to prove that \vspace{0.2cm}

${(1+h)}^{k+1} \ge 1+ (k+1)h$. We have \vspace{0.2cm}

\hspace{1cm} ${(1+h)}^{k+1} = {(1+h)}^k(1+h) $ \vspace{0.2cm}

\hspace{2cm} $\ge (1+kh)(1+h)$ \hspace{1.5cm} (by inductive hyp. and $1+h > 0$) \vspace{0.2cm}

\hspace{2cm} $= 1 + h + kh + kh^2$ \vspace{0.2cm}

\hspace{2cm} $\ge 1 + h + kh +0$ \vspace{0.2cm}

\hspace{2cm} $= 1+ (k+1)h$ \vspace{0.2cm}

so the inequality is true for $n=k+1$ if it is true for $n=k$. By the principle of  \vspace{0.2cm}

mathematical induction, \vspace{0.2cm} 

\hspace{2cm} ${1+h}^n \ge 1+nh$ for all $n \in \field{N}$. \vspace{1cm}

Sometimes it is convenient to use the principle of mathematical induction \vspace{0.2cm}

in a slightly different form. We assume (not only $q_k$, but rather) each \vspace{0.2cm}

one of $q_1$, $q_2$, $q_3$, \dots, $q_k$ is true and then conclude that $q_{k+1}$ is true. This \vspace{0.2cm}

establishes the truth of $q_n$ for all  $n \in \field{N}$, as the following lemma shows. \vspace{1cm}

\textbf{4.4 \hspace{0.5cm} Lemma:}\textit{  Let $q_n$ be a statement involving a natural number $n$. Assume that}  \vspace{0.2cm}

\hspace{2cm}  \textbf{i.}\textit{ $q_1$ is true, } \vspace{0.2cm}

\hspace{2cm} \textbf{ii.}\textit{ for all  $k \in \field{N}$, if  $q_1$, $q_2$, $q_3$, \dots, $q_k$ are true, then $q_{k+1}$ is true. } \vspace{0.2cm}

\textit{ Then $q_n$ is true for all  $n \in \field{N}$. }































\end{document}

