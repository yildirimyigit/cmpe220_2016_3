\documentclass[11pts]{amsbook}
\usepackage[utf8]{inputenc}
\usepackage{HBSuerDemir}
\usepackage[utf8]{inputenc}
\usepackage{amsfonts}
\usepackage{natbib}
\usepackage{graphicx}
\usepackage{enumitem} 

\begin{document}
\hPage{feyzioglu-030}
\begin{center}
\textbf{{\LARGE{\S4}}} \\
\textbf{{\LARGE{Mathematical Induction}}} 
\end{center}
\noindent Examine the propositions
\begin{center}
$2^{n} \geq n$ for all $n \epsilon \mathbb{N}$, \\
$1+2+...+n=\frac{n(n+1)}{2} $ for all $n \epsilon \mathbb{N}$, \\
$n^{n+1} \geq (n+1)^{n}$ for all $n \epsilon \mathbb{N}$. \\
\end{center}
How do we prove them? They are statements involving a variable \textit{n} running through the infinite set $\mathbb{N}$. Strictly speaking; each one of these propositions above is a collection of infinitely many, propositions. We can verify them for a finite number of cases where \textit{n} assumes some specific values. Thus we might verify $2^{n}\geq n$ for n=1,2,3, ... , 1 000 000 and convince ourselves -of the truth of this statement, but this is far from a proof. On the other hand, we cannot check the truth of infinitely many statements within finite time. So we must resort to some other means.\\
\newline
In order to prove propositions about all natural numbers, an axiom is introduced. It is the fifth Peano axiom about $\mathbb{N}$ Giuseppe Peano (1858-1932), an Italian mathematician and logician). It is called the axiom of mathematical induction.\\
\newline
\textbf{4.1 Axiom (of mathematical induction):} If S is a subset of $\mathbb{N}$ such that \\

\begin{enumerate}[label=(\roman*)]
\item 1 $\epsilon$ S
\item for all k $\epsilon \mathbb{N}$, if k $\epsilon$ S, then k+1  $\epsilon \mathbb{N}$,
\end{enumerate}
then S is the whole of $\mathbb{N}$, i.e., S=$\mathbb{N}$. \\
\newline
We can use this axiom to prove statements of the form '$p_{n}$ for all $n \epsilon \mathbb{N}$' as follows. We let S $\subseteq \mathbb{N}$ be the set of all natural numbers \textit{n} for which $p_{n}$ is true. First we verify 1 $\epsilon$ S, that is, we verify that $p_{1}$ is true. Second, we \textit{assume} that 
 k $\epsilon$ S and under this hypothesis, which is called the  \textit{induction hypothesis}, we prove that $p_{k+1}$ is true. So we show that k $\epsilon$ S implies  k+1 $\epsilon$ S. By the axiom of mathematical induction, $S=\mathbb{N}$, so the
\end{document}
