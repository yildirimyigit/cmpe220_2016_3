\documentclass[11 pt]{amsbook}
\usepackage{../HBSuerDemir}

\begin{document}
\hPage{feyzioglu-092}

\noindent In the second place, we must check associativity. For all $a,b,c \in H$, we must show $(ab)c = a(bc)$. But we know that $(ab)c = a(bc)$ for all $a,b,c \in G$. Since $H \nsubseteq G$, we have all the more so $(ab)c = d(bc)$ for all $a,b,c \in H$. Indeed, if all the elements of $G$ have a certain property, then all the elements of $H$ will have the same property. Thus associativity holds in $H$ automatically, so to speak. We do not have to check it. \\

\noindent In $H$, there must exist an identity, say $1_H \in H$ such that $al_H = a$ for all $a \in H$. In particular, the identity $1_H$ of $H$ has to be such that $1_H 1_H = 1_H$ Since $1_H \in H \insubseteq G$, Lemma 7.3(1) yields $1_H = 1_H =$ identity element of $G$. So the identity element of $G$ is also the identity element of $H, provided$ it belongs to $H$. Then we do not have to look for an identity element of $H$, we must only check that the identity element of $G$ does belong to $H$. We write $1$ for the identity element of $H$, since it is the identity element of $G$. \\

\noindent Finally, for each $a \in H$, there must exist an $x \in H$ such that $ax = 1$. Reading this equation in $G$, we see $x = a^{-1} =$ the inverse of $a$ in $G$. We know that the inverse of $a$ exists. Where? The inverse of $a$ exists in $G$. We must also check $a^{-1} \in H$. Thus we do not have to look for an inverse of $a$. We must only check that the inverse $a^{-1}$ of $a$, which we know to be in $G$, is in fact an element of $H$. \\

\noindent Summarizing this discussion, we see that a non empty subset $H$ of a group $G$ is a subgroup of $G$ if and only if
\begin{itemize}
    \item [1)] $ab \in H$ for all $a,b \in H$,
    \item [2)] $1 \in H$
    \item [3)] $a^{-1} \in H$ for all $a \in H$
\end{itemize}

\noindent Moreover, (2) follows from (1) and (3). Indeed, if $a \in H$ (remember that $H \neq \emptyset$), then $a^{-1} \in H$ by (3) and hence $aa^{-1} \in H$ by (1), which gives $1 \in H$. So (1),(2),(3) together is equivalent to (1),(3) together. We proved the following lemma. \\ \\

\subsection*{9.2 Lemma (Subgroup criterion: \) } \textit{Let G be a group and let H be a nonempty subset of G Then H is a subgroup of G if and only if}
\begin{itemize}
    \item[(i)] for all $a,b \in H$, we have $ab \in H$ (H is closed under multiplication) and
\end{itemize}

\end{document}
